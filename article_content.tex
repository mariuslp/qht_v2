\section{Introduction}
\subsection{Motivation}

Throughout this paper, we are interested in the following problem: 
\begin{definition}[Duplicate detection problem over a sliding window, wDDP] Given a stream $E_n = (e_1, e_2, \dotsc, e_n)$, a sliding window size $w$ and a \enquote{new} item $e^\star$, find whether $e^\star$ is also present in the last $w$ elements of the stream, ie., whether $e^\star \in \{e_{n-w+1}, \dotsc, e_n\}$.
At every time increment, the new item is added to the stream, i.e., $E_{n+1} = E_n \mid e^\star$ where $\mid$ denotes concatenation. 
\end{definition}

Note that for $w = \infty$, the problem becomes finding whether an element is a duplicate amongst all previous stream elements. For simplicity in the notation, when we refer to $\infty$DDP we instead write DDP.

Instances of the wDDP abound in computer science, with applications to file system indexation, database queries, network load balancing, network management \cite{10.5555/647912.740658}, in credit card fraud detection \cite{DBLP:journals/corr/abs-1709-08920}, phone calls data mining \cite{10.1145/347090.347094}, etc. A discussion about algorithms on large data streams can be found in \cite{10.1145/776985.776986}. 

In practice, additional constraints exist that we can capture with the following definition:
\begin{definition}[wDDP with bounded memory]
	At every time step $n$, given $e^{\star}$ and a current state (dependent on history) of at most $M$ bits, solve the wDDP for $E_n$ and $e^{\star}$.
\end{definition}
%Even when the stream is bounded, i.e. $\lim_{n\to\infty}|E_n| < \infty$, it may be impractical to perform a complete lookup and approximate solutions are desirable. 
%Perfect detection of duplicates in a stream with $U$ possible values requires $U$ bits of memory \cite[Theorem 2.1]{10.1145/3297280.3297335}: even for modest situations (e.g. 64-bit nonces, corresponding to $2^{64}$ possible values) this is quickly impractical. As a result, we need a further relaxation of the DDP that allows for errors: 
Perfect detection is however not always reachable and it might be more practical to work on a further relaxation of the problem, allowing for errors.

%\begin{definition}[Approximate wDDP with bounded memory]
%	Solve the wDDP using at most $M$ bits of memory, with at most $\epsilon < 1/2$ mispredictions.\footnote{Given a classifier with $\epsilon > 1/2$, we obtain a classifier with $\epsilon < 1/2$ by reversing its output.} 
%\end{definition}

Approximate duplicate detection has many real-life use cases, and can sometimes play a critical role, for instance in cryptographic schemes where all security and secrecy fall apart as soon as a random nonce is used twice, such as the ElGamal \cite{10.1007/3-540-39568-7_2} or ECDSA signatures. Other uses include improvements over caches \cite{4484874}, duplicate clicks \cite{10.1145/1060745.1060753}, and others.

On a side note, it is clear that the input distribution plays a central role regarding how efficiently the wDDP can be solved. For instance, some deterministic streams may be expressed very compactly (such as the output of a PRNG with known seed) making the wDDP relatively easy. Information-theoretically, if the source has $U$ bits of entropy then the situation is equivalent to having an $U$-bit, uniformly distributed input. This is the setting we consider here.

As said before, when the window size in wDDP grows infinitely large, it becomes the following problem: find whether $e^\star \in E_n$. %This \enquote{non-windowed} problem predates in fact the definition of wDDP, and several algorithms have been proposed for this problem: \cite{Den06,10.1145/3297280.3297335,Dut13} and a slight variation of Cuckoo Filters \cite{Fan14} suggested by \cite{10.1145/3297280.3297335}.
Unfortunately any solution to this problem will necessary encounter a phenomenon called \enquote{saturation} on large enough data streams \cite{10.1145/3297280.3297335}, and when it happens the algorithm performs no better than at random. 

This is problematic on two grounds: it makes the comparison of several algorithms difficult (since they all asymptotically behave in that fashion), and the unavoidable saturation ruins any particular design's merits. As such, it is more interesting to focus on wDDP rather than DDP.

\subsection{Contributions}
In this paper, we start from a naïve solution for the wDDP to then derive bounds for when it can be solved within $M$ bits of memory, up to a window size $w_\text{max}$, in constant time. We then introduce a generalization of the naïve solution, and study its error rate. We show that this construction, which we call Short Hash Filter (SHF), can push the value $w_\text{max}$ further while operating in constant time --- at the cost of some errors. %We also provide a different tradeoff, the Compact Short Hash Filter (CSHF), which uses fewer memory but operates in linear time.

Unfortunately, for $w > w_\text{max}$ the performance of SHF degrades very rapidly. We therefore turn our attention to existing data structures designed for the \enquote{non-windowed} setting. We show that some of them outperform dedicated data structures, including SHF, in the $w > w_\text{max}$ regime.

We then introduce the \enquote{queuing construction}, a black box transformation of non-windowed data structures into windowed ones, that improves their performance in the wDDP setting.

Finally, we provide an analysis of our queueing construction's resistance to adversarial streams.

\subsection{Related work}\label{sub:related}
The notion of sliding window was, as far as we know, first introduced in \cite{10.1145/1060745.1060753}.%; but several variations exist that are incomparable to one another (e.g. \cite{shtul2020agepartitioned}).
The wDDP formulation we rely on is due to \cite{Yoo10,She08}, which also introduce algorithms for solving the wDDP approximately.

The notion of using subfilters, as in the queuing construction, can be found in the A2 filter's design \cite{Yoo10} and a variation thereof can be found in \cite{shtul2020agepartitioned} but in a different DDP formulation. The A2 is built from two Bloom filters, a construction which we generalize and analyse generically in this paper. Similarly, the construction in \cite{shtul2020agepartitioned} only works with Bloom Filters.

%\section{Duplicate Detection and Filter Saturation}
\label{sec:dudefisa}
\section{Notations and basic definitions}
We consider an unbounded stream $E = (e_1, e_2, \dotsc, e_n, \dotsc)$ with elements belonging to an alphabet $\Gamma$.

%A filter is an algorithm, which has a finite amount of memory $M$ and, for each new element $e$, outputs $\DUPLICATE$ or $\UNSEEN$ whether it thinks $e$ is a duplicate or not.;

We usually consider the situations where the available memory is too small for perfect detection, i.e., $M \ll |\Gamma|$. Otherwise, if $M = |\Gamma|$ then the problem can be solved in constant time without errors \cite{10.1145/3297280.3297335}.

An element $e_j$ is a duplicate in $E$ over the sliding window $w$, and we note $e_i \wdup E$ if there exists $j-w \leq i < j$ such that $e_i = e_j$. Otherwise we note $e_i \notwdup E$, and we say $e_i$ is unseen over $w$.
A false positive over $w$ is an element $e\notwdup E$ which is classified as a duplicate, and a false negative is an element $e\wdup E$ which is classified as a not-duplicate.

For a filter, the probability of false positive ($\FP^w_i$) is the probability that after $i$ insertions, the \emph{unseen} element $e_i\notwdup E$ is a false positive over $w$. The false positive rate $\FPR^w_i$ is the number of false positives divided by the number of unseen elements in E\footnote{We observe that $\FPR^w_n = \frac{1}{n}\sum_{i=1}^n \FP^w_i$, and similarly for $\FNR^w_n$.}. We similarly define the false negative probability $\FN^w_i$, and the false negative rate $\FNR^w_i$.



\paragraph{Remark.} For benchmarking, we usually measure the error rate $ER = \FPR^w + \FNR^w$, as it allows a practical ranking of the solutions. An error rate of $0$ means a perfect filter, while a filter answering randomly has an error rate of $1$. A filter being always wrong has an error rate of $2$.


%\section{DDFs and the wDDP}
%A literature review collects the following DDF constructions: A2 filters \cite{Yoo10}, Stable Bloom Filters (SBF) \cite{Den06}, Quotient Hash Tables (QHT) \cite{10.1145/3297280.3297335}, Streaming Quotient Filters (SQF) \cite{Dut13}, Block-decaying Bloom Filters (b\_DBF) \cite{She08}, and a slight variation of Cuckoo Filters \cite{Fan14} suggested by \cite{10.1145/3297280.3297335}. The structure proposed in \cite{10.1145/1060745.1060753} is not desined for wDDP but a variant called `landmark` sliding window, which consists of a zero-resetting of the memory at some user-defined epochs.
%
%Of these, only the A2 and b\_DBF were designed explicitly for the wDDP. Others DDFs do not account for a finite-length sliding window, and therefore cannot be adjusted as a function of $w$. Another remarkable (but experimental) observation is that these DDFs' false positive rate does not decrease to 0 when the sliding window becomes small.


\section{Approximate solution and SHF}
%\subsection{Optimal wDDF}\label{sub:opt_solutions}
%
%Our goal is to minimise both the false positive and false negative rates of a DDF, given a fixed and limited amount of memory. Before going into details, we first explore the limits of the problem, i.e., when perfect or very performant solutions exist.
%
%\begin{theorem}
%	For $M \geq w (\log_2(w) + 2\log_2(|\Gamma|))$, the wDDP can be solved exactly (with no errors) in constant time.
%\end{theorem}

%\begin{proof}
%	We explicitly construct a DDF that performs the detection. Storing all $w$ elements in the sliding window takes $w\log_2(|\Gamma|)$ memory, using a FIFO queue $Q$; however 
%	lookup has a worst-time complexity of $O(w)$. 
%	
%	We therefore rely on an
%	ancillary data structure for the sake of quickly answering lookup questions.
%	Namely we use a dictionary $D$ whose keys are elements from $\Gamma$ and values are counters.
%	
%	When an element $e$ is inserted in the DDF, $e$ is stored and $D[e]$ is incremented (if the key $e$ did not exist in $D$, it is created first, and
%	$D[e]$ is set to $1$). In order to keep the number of stored elements to $w$, 
%	we discard the oldest element $e_\text{last}$ in $Q$. As we do so, we also decrement $D[e_\text{last}]$, and if $D[e_{last}] = 0$ the key is deleted from $D$. The whole insertion procedure is therefore performed in constant time.
%	
%	Lookup of an element $e^\star$ is simply done by looking whether the key $D[e^\star]$ exists, which is done in constant time.
%	
%	The size of the queue is $w\log_2 |\Gamma|$, the size of the dictionary is $w (\log_2 |\Gamma| + \log_2 w)$ (as the dictionary cannot have more than $w$ keys at the same time, a dictionary key occupies $\log_2 |\Gamma|$ bits and a counter cannot go over $w$, thus being less than $\log_2 w$ bits long). Thus a requirement of $w (\log_2(w) + 2\log_2(|\Gamma|))$ bits for this DDF to work.
%
%	Finally this filter does not make any mistake, as the dictionary $D$ keeps an exact account of how many times each element is present in the sliding window.
%\end{proof}

\subsection{Optimal and Approximate Optimal wDDF}
%The optimal filter described above requires that the size of $\Gamma$ is known in advance. The dependence on $\log_2 |\Gamma|$ can be dropped, at the cost of allowing errors.
\begin{restatable}{thm}{optwddf}\label{thm:opt_wddf}
	For $M \geq w (\log_2(w) + 2\log_2(|\Gamma|))$, the wDDP can be solved exactly (with no errors) in constant time.
\end{restatable}
	
\begin{proof}
	Due to page limit restrictions, all theorem proofs' are in Appendix~\ref{app:proofs}.
\end{proof}

However, this optimal filter requires that the size of $\Gamma$ is known in advance. The dependence on $\log_2 |\Gamma|$ can be dropped, at the cost of allowing errors.

\begin{restatable}{thm}{shf}\label{thm:SHF}
	Let $w \in \mathbb N$. Let $M \simeq 2w\log_2w$, then the wDDP can be solved with almost no error using $M$ bits of memory.
	
	More precisely, it is possible to create a filter of $M$ bits with an FN of $0$, an FP of $1 - (1-\frac1{w^2})^w \sim \frac 1w$, and a time complexity of $O(w)$.
	
	Using $M \simeq 5w\log_2 w$ bits of memory, a constant-time filter with the same error rate can be constructed.
\end{restatable}
Note that we only consider the probability of false positive after the filter has inserted at least $w$ elements, i.e., once the filter is full and has reached a stationary regime.


%\begin{proof}
%	Here again we explicitly construct the filters that attain the theorem's bounds.
%
%	Let $h$ be a hash function with codomain $\{0,1\}^{2 \log_2 w}$.
%The birthday theorem \cite{10.1007/3-540-45708-9_19} states that for a hash function $h$ over $a$ bits, one must on average collect $2^{a/2}$ input-output pairs before obtaining a collision. Therefore $2^{(2 \log_2 w) / 2} = w$ hash values $h(e_i)$ can be computed before having a $50\%$ probability of a collision (here, a collision is when two distinct elements of the stream $e_i, e_j$ with $i \neq j, e_i \neq e_j$ have the same hash, i.e. $h(e_i) = h(e_j)$). The 50\% threshold we impose on $h$ is arbitrary but nonetheless practical.
%
%Let $\mathcal F$ be the following DDF: the filter's state consists in a queue of $w$ hashes, and for each new element $e$, $\mathsf{Detect}(e)$ returns \DUPLICATE if $h(e)$ is present in the queue, \UNSEEN otherwise. $\mathsf{Insert}(e)$ appends $h(e)$ to the queue before popping the queue. 
%
%There is no false negative, and a false positive only happens if the new element to be inserted collides with at least one other element, which happens with probability $1 - (1 - \frac 1{2^{2\log_2w}})^{w} = 1 - (1-\frac1{w^2})^w$, hence an FN of $0$ and a FP of $1 - (1-\frac1{w^2})^w$.
%The queue stores $w$ hashes, and as such requires $w \cdot 2\log_2 w$ bits of memory.
%
%Note that this solution has a time complexity of $O(w)$. Using an additional dictionary, as in the previous proof, but with keys of size $2\log_2(w)$, we get a filter with an error rate of about $\frac 1w$ and constant time for insertion and lookup, using $w \cdot 2\log_2 w + w \cdot (2\log_2(w) + \log_2(w)) = 5w\log_2 w$ bits of memory.
%\end{proof}

When $\log_2|\Gamma| > 5 \log_2 w$ this DDF outperforms the naïve strategy\footnote{The naïve strategy consisting of storing the $w$ elements of the sliding window, requiring $w \log_2|\Gamma|$ bits of memory.}, both in terms of time and memory, at the cost of a minimal error. When $\log_2|\Gamma| > 2 \log_2 w$, it outperforms the exact solution described sooner in terms of memory.

%\section{Short Hash Filter}
\subsection{Short Hash Filter Algorithm}
The approximate filter we described uses hashes of size $2\log_2(w)$ for a given sliding window $w$. However, this hash size is arbitrary, and while it guarantees a very low error rate, it can be changed. More importantly, in some practical cases the maximal amount of available memory is fixed beforehand. Fixing the memory is also more practical for benchmarking data structures, as it gives the guarantee that all filters operate under the same conditions.

This gives us the Short Hash Filter (SHF), described in Algorithm~\ref{alg:SHF}. The implementation relies on a double-ended queue or a ring buffer, which allows pushing at beggining of a queue and popping at the end in constant time.


\begin{algorithm}
	\caption{SHF Setup, Lookup and Insert}\label{alg:SHF}
	
	\begin{algorithmic}[1]
	\Function{Setup}{$M, w$}
	\Comment $M$ is the available memory, $w$ the size of the sliding window
	\State $h \gets$ hash function of codomain size $\lfloor \frac M{2w} - \frac 12 \log_2 w \rfloor$
	\State $Q \gets \emptyset$ \Comment $Q$ is a queue of elements of size $h$
	\State $D \gets \emptyset$ \Comment $D$ is a dictionary $h \Rightarrow$  counter (of max value $w$)
	
%	\State\Return filter
	\EndFunction 
	\end{algorithmic}

	\begin{multicols}{2}
	\begin{algorithmic}[1]
	\item[]
	\item[]
	\Function{Lookup}{$e$}
	\If{$D[h(e)] > 0$}
	\State \Return \DUPLICATE
	\EndIf
	\State \Return \UNSEEN
	\EndFunction
	\item[]
	\item[]
\end{algorithmic}

\begin{algorithmic}[1]
	\Function{Insert}{$e$}
	\State $Q.\mathsf{Push\_Front}(h(e))$
	\State $D[h(e)]$++
	
	\If{$Q.\mathsf{length}() > w$}
	\State $h' \gets Q.\mathsf{Pop\_back}()$
	\State $D[h']$-{}-
	\If{$D[h'] = 0$}
	\State Erase key $D[h']$
	\EndIf
	\EndIf
	\EndFunction
\end{algorithmic}
	\end{multicols}
\end{algorithm}

%\subsection{Compact Short Hash Filter}
%Removing the dictionary from the SHF construction yields a more memory-efficient, but less time-efficient construction, which we dub \enquote{compact} short hash filter (CSHF). The CSHF performs in linear time in $w$, and is a simple queue, the only point is that instead of storing $e$, one stores $h(e)$, where $h$ is a hash function of codomain size $\lfloor \frac Mw \rfloor$.% and is described in Algorithm~\ref{alg:CSHF}.

%\begin{algorithm}[!h]
%	\caption{CSHF, Setup, Lookup and Insert}\label{alg:CSHF}
%	
%	\begin{algorithmic}[1]
%		\Function{Setup}{$\mathcal M, w$}
%		\Comment $M$ is the available memory, $w$ the size of the sliding window
%		\State $h \gets$ hash function of codomain size $\lfloor \frac Mw \rfloor$
%		\State $Q \gets \emptyset$ \Comment $Q$ is a queue of elements of size $h$
%		\State\Return filter
%		\EndFunction 
%		\item[]
%	\end{algorithmic}
%	
%	\begin{algorithmic}[1]
%		\Function{Lookup}{$e$}
%		\State \Return $h(e) \in Q$ 
%		\EndFunction
%		\item[]
%	\end{algorithmic}
%	
%	\begin{algorithmic}[1]
%		\Function{Insert}{$e$}
%		\State $Q.\mathsf{Push\_Front}(h(e))$
%		
%		\If{$Q.\mathsf{length}() > w$}
%			\State $h' \gets Q.\mathsf{Pop\_back}()$
%		\EndIf
%		\EndFunction
%	\end{algorithmic}
%\end{algorithm}

\paragraph{Error Rates.}
Let $w> 0$ be a window size and $M > 0$ the available memory.

%We write $\FN^w_\text{SHF}$ the probability of false negative of an SHF with these parameters. We similarly define $\FP^w_\text{SHF}$.%, $\FN^w_\text{CSHF}$, $\FP^w_\text{CSHF}$. 

\begin{restatable}{thm}{fprSHF}\label{thm:fprSHF}	
	$\FN_\text{SHF}^w  = 0$  and $\FP_\text{SHF}^w = 1 - \left(1 - \sqrt{w2^{-M/w}}\right)^w$
\end{restatable}

%\begin{proof}
%	This is an immediate adaptation of the proof from Theorem~\ref{thm:SHF}. An SHF has fingerprints of size $h = \frac M{2w} - \frac 12 \log_2 w$, while a CSHF has fingerprints of size $h' = \frac Mw$.
%\end{proof}

%\paragraph{Remark:} A CSHF of size $M$ has the same error rate than an SHF of size $2M + w\log_2 w$.
\paragraph{Saturation.}
SHF and have strictly increasing error rates, which reach a threshold of $1/2$ for some maximum window size $w_\text{max}$. Beyond this value, these filters saturate extremely quickly: in other words, most SHF will either have an error rate of $0$ or $1$.

An illustration of this phenomenon can be seen in Figure~\ref{fig:shorts}, 
which shows the error rates for SHF with $M=10^5$, against 
a uniformly random stream of $18$-bit elements ($|\Gamma| = 2^{18}$). The benchmark used a finite stream of length $10^6$.

\begin{figure}[t]
\makebox[\textwidth]{\makebox[1.05\textwidth]{%
\begin{minipage}{.5\textwidth}
	\centering
	% Stream length = 100000
\begin{filecontents*}[overwrite]{graphs/CSHF.dat}
	a	b
	100 	0.802639224503
	115 	0.492910959288
	133 	1.31598958014
	153 	0.335308554416
	177 	0.438456547114
	205 	0.476590812473
	237 	0.540940910883
	273 	0.196587101327
	316 	0.246100734697
	365 	0.217151187997
	421 	0.128641911778
	486 	0.0535639648759
	562 	0.0935842266715
	649 	0.12093844298
	749 	0.170342037841
	865 	0.0308566937748
	1000 	0.051849837015
	1154 	0.0686718679939
	1333 	0.0598027993296
	1539 	0.0849902359533
	1778 	0.0905799179135
	2053 	0.0518288188455
	2371 	0.0457712639288
	2738 	0.0391151025132
	3162 	0.0423725009537
	3651 	0.017465422709
	4216 	0.0548122676583
	4869 	0.473534801299
	5623 	4.15215864652
	6493 	17.7693433112
	7498 	59.2675832042
	8659 	98.2648288551
	10000 	99.8860823115
	11547 	99.9732521983
	13335 	99.9865373268
	15399 	99.9932155525
	17782 	99.9965788171
	20535 	99.9982713131
	23713 	99.9982507716
	27384 	99.9991131735
	31622 	99.9990992604
	36517 	99.9995414232
	42169 	99.9995318533
	48696 	99.999520767
	56234 	99.9997536682
	64938 	99.9997461729
	74989 	99.9997368608
	86596 	99.9997263681
	100000 	99.999713481
	
\end{filecontents*}
\begin{filecontents*}[overwrite]{graphs/SHF.dat}
	w	 FPR 	 FNR 	 Error 	 1-E 	 Phi 	 percentDup
100 	0 	0 	0 	100 	100 	0.04
112 	0 	0 	0 	100 	100 	0.05
125 	0 	0 	0 	100 	100 	0.05
141 	0 	0 	0 	100 	100 	0.06
158 	0 	0 	0 	100 	100 	0.06
177 	0 	0 	0 	100 	100 	0.07
199 	0 	0 	0 	100 	100 	0.08
223 	0 	0 	0 	100 	100 	0.09
251 	0 	0 	0 	100 	100 	0.1
281 	0 	0 	0 	100 	100 	0.11
316 	0 	0 	0 	100 	100 	0.12
354 	0 	0 	0 	100 	100 	0.14
398 	0 	0 	0 	100 	100 	0.16
446 	0 	0 	0 	100 	100 	0.18
501 	0 	0 	0 	100 	100 	0.2
562 	0 	0 	0 	100 	100 	0.23
630 	0 	0 	0 	100 	100 	0.25
707 	0 	0 	0 	100 	100 	0.28
794 	0 	0 	0 	100 	100 	0.31
891 	0 	0 	0 	100 	100 	0.34
999 	0 	0 	0 	100 	100 	0.38
1122 	0 	0 	0 	100 	100 	0.43
1258 	0 	0 	0 	100 	99.99 	0.47
1412 	0 	0 	0 	100 	99.97 	0.53
1584 	0 	0 	0 	100 	99.8 	0.6
1778 	0.04 	0 	0.04 	99.96 	97.06 	0.68
1995 	0.38 	0 	0.38 	99.62 	81.7 	0.76
2238 	3.34 	0 	3.34 	96.66 	44.51 	0.85
2511 	14.1 	0 	14.1 	85.9 	23.38 	0.95
2818 	49.49 	0 	49.49 	50.51 	10.38 	1.07
3162 	99.74 	0 	99.74 	0.26 	0.56 	1.2
3548 	99.97 	0 	99.97 	0.03 	0.19 	1.36
3981 	99.99 	0 	99.99 	0.01 	0.1 	1.51
4466 	100 	0 	100 	0 	0.07 	1.7
5011 	100 	0 	100 	0 	0.04 	1.91
5623 	100 	0 	100 	0 	0.03 	2.14
6309 	100 	0 	100 	0 	0.02 	2.39
7079 	100 	0 	100 	0 	0.02 	2.67
7943 	100 	0 	100 	0 	0.02 	2.99
8912 	100 	0 	100 	0 	0.02 	3.35
9999 	100 	0 	100 	0 	0.02 	3.74
11220 	100 	0 	100 	0 	0.02 	4.18
12589 	100 	0 	100 	0 	0.02 	4.68
14125 	100 	0 	100 	0 	0.02 	5.22
15848 	100 	0 	100 	0 	0.02 	5.84
17782 	100 	0 	100 	0 	0.03 	6.52
19952 	100 	0 	100 	0 	0.03 	7.26
22387 	100 	0 	100 	0 	0.03 	8.1
25118 	100 	0 	100 	0 	0.03 	9.01
28183 	100 	0 	100 	0 	0.03 	10.04
31622 	100 	0 	100 	0 	0.04 	11.19
35481 	100 	0 	100 	0 	0.04 	12.43
39810 	100 	0 	100 	0 	0.04 	13.82
44668 	100 	0 	100 	0 	0.04 	15.34
50118 	100 	0 	100 	0 	0.05 	17
56234 	100 	0 	100 	0 	0.05 	18.81
63095 	100 	0 	100 	0 	0.05 	20.73
70794 	100 	0 	100 	0 	0.05 	22.84
79432 	100 	0 	100 	0 	0.06 	25.15
89125 	100 	0 	100 	0 	0.06 	27.58
\end{filecontents*}
\begin{tikzpicture}
\begin{semilogxaxis}[width=0.9\textwidth, 
	%legend pos = outer north east, 
	legend style={at={(0.02,0.98)},anchor=north west},%
	xlabel=$w$, ylabel=$\FPR^w+\FNR^w (\times 100)$]

\addplot[color=blue, mark=o] table[x=w,y=Error] {graphs/SHF.dat};
\addlegendentry{SHF}

%\addplot[color=red, mark=x] table[x=a,y=b] {graphs/CSHF.dat};
%\addlegendentry{CSHF}

\end{semilogxaxis}
\end{tikzpicture}
	\caption{Error rates of SHFs and CSHFs for $M = 10^5$ bits, for varying window sizes $w$.}\label{fig:shorts}
\end{minipage}
\hfill
\begin{minipage}{.5\textwidth}
	\centering
	\begin{tikzpicture}
		%\begin{tikzpicture}
\begin{axis}[
	width=0.99\textwidth,
	xmode=log,
	xmin = 500,
	xmax = 200000000,
%	extra x ticks = {500},
	xlabel = {Size of the stream},
	ymax = 110,
    ymin = 0,
    ytick = {0, 20, ..., 120},
    ylabel = {$\FPR^\infty + \FNR^\infty (\times 100)$},
%    title = Artificial Stream,
	%legend pos = outer north east,
	legend style={at={(0.02,0.98)},anchor=north west,nodes={scale=0.9, transform shape}},%
    %width = 0.4\textwidth
]
%\nextgroupplot[
%xmode=log,
%xmin = 500,
%xmax = 350000000,
%%	extra x ticks = {500},
%xlabel = {Size of the stream},
%ymax = 110,
%ymin = 0,
%ytick = {0, 20, ..., 120},
%%    ylabel = {FPR + FNR},
%title = Uniform stream,
%    legend columns=1,
%    legend to name= legend_n,
%]
%% 1 QHT 
\addplot [color=red, mark=+] coordinates {
	(1000, 0)
	(10000, 0.28)
	(30000, 0.67)
	(50000, 5.55)
	(100000, 8.33)
	(300000, 24.07)
	(1000000, 56.52)
	(3000000, 81.41)
	(10000000, 93.74)
	(50000000, 98.83)
	(100000000, 99.03)
	(150000000, 99.21)
};
\addlegendentry{QHT}


%% 2 SQF
\addplot[color=blue, mark=x] coordinates {
	(1000, 0)
	(10000, 0.49)
	(30000, 1.38)
	(50000, 6.67)
	(100000, 10.4)
	(300000, 33.54)
	(1000000, 64.9)
	(3000000, 85.91)
	(10000000, 95.24)
	(50000000, 98.44)
	(100000000, 99.27)
	(150000000, 99.40)
};
\addlegendentry{SQF}

%% 3 Cuckoo
\addplot [color=teal, mark=triangle] coordinates {
	(1000, 0.2)
	(10000, 2.48)
	(30000, 6.63)
	(50000, 19.81)
	(100000, 28.8)
	(300000, 72.17)
	(1000000, 90.88)
	(3000000, 96.28)
	(10000000, 98.95)
	(50000000, 99.75)
	(100000000, 99.84)
	(150000000, 99.86)
};
\addlegendentry{Cuckoo}

%% 4 SBF 
\addplot [color=olive, mark=square, mark options = {scale=0.7}]coordinates {
	(1000, 0.3)
	(10000, 4.07)
	(30000, 10.18)
	(50000, 14.63)
	(100000, 46.64)
	(300000, 77.46)
	(1000000, 92.66)
	(3000000, 97.22)
	(10000000, 99.19)
	(50000000, 99.79)
	(100000000, 99.88)
	(150000000, 99.90)
};
\addlegendentry{SBF}

%%% 5 A2 
%\addplot [color=orange, mark=diamond]coordinates {
%	(1000, 0)
%	(10000, 0.03)
%	(30000, 0.38)
%	(50000, 1.1)
%	(100000, 3.92)
%	(300000, 19.6)
%	(1000000, 63.3)
%	(3000000, 86.66)
%	(10000000, 95.67)
%	(50000000, 98.94)
%	(100000000, 99.33)
%	(150000000, 99.46)
%};
%\addlegendentry{A2}
%
%
%% 5 bDBF 
%\addplot [color=magenta, mark=pentagon]coordinates {
%	(1000, 0)
%	(10000, 0.12)
%	(30000, 60.13)
%	(50000, 81.96)
%	(100000, 91.3)
%	(300000, 96.71)
%	(1000000, 98.94)
%	(3000000, 99.57)
%	(10000000, 99.87)
%	(50000000, 99.97)
%	(100000000, 99.98)
%	(150000000, 99.99)
%};
%\addlegendentry{b\_DBF}


\addplot[black, pattern=north east lines wide] coordinates
{
	(1000000, 0)
	(2000000, 49.63)
	(3000000, 66.16)
	(5000000, 79.40)
	(10000000, 89.31)
	(30000000, 95.90)
	(50000000, 97.18)
	(100000000, 98.09)
	(150000000, 98.34)
	(300000000, 98.50)
} \closedcycle; 

\end{axis}
%\end{tikzpicture}
	\end{tikzpicture}
	\caption{Error rate (times 100) of DDFs of 1Mb as a function of stream length. Hatched area represents over-optimal (impossible) values.}\label{fig:graph_n}
\end{minipage}
}}
\end{figure}

The value $w_\text{max}$ can be obtained by solving (numerically) for $\FP^{w_\text{max}} = 1/2$ for a given $M$. %We numerically solved the equation $\FP^w = 0.5$ for these two filters, for about 200 different values of $M$, uniformly distributed \emph{on a log scale} between $10^2$ and $10^6$. Results
% are given in Figure~\ref{fig:w_opt} and 
Experiments indicate an approximately linear relationship between $M$ and $w_\text{max}$: %for large values of $M$:
 %$w_\text{max}^\text{CSHF} = 0.0627 M + 443$ ($r^2 = 0.9981$) and
$w_\text{max}^\text{SHF} = 0.0233 M + 186$ ($r^2 = 0.9977$).

%\begin{figure}[]
%	\begin{filecontents*}[overwrite]{w_SHF.dat}
		a	b
100.0 9.26213732804 0.500276581119
105.0 9.59232049452 0.500067042032
110.0 9.91954451157 0.499920330905
115.0 10.2466796875 0.500333250323
121.0 10.6357531474 0.500853531439
127.0 11.0141300273 0.500131062355
134.0 11.4554296875 0.500093704401
140.0 11.8343496531 0.500788989541
147.0 12.265553627 0.500434592551
155.0 12.7506201386 0.499607550218
162.0 13.1715771588 0.499011238342
171.0 13.7249623798 0.500890655812
179.0 14.2011946474 0.500851009461
188.0 14.7242361306 0.499690060252
197.0 15.2542726359 0.500227342286
207.0 15.8343979408 0.500321874068
218.0 16.4656073228 0.500304978866
229.0 17.0882780223 0.499958414774
240.0 17.7096679688 0.500158138473
252.0 18.3743968508 0.499501219123
265.0 19.1027941549 0.500555714764
278.0 19.8167115001 0.500638447418
292.0 20.5774858827 0.500575709642
307.0 21.3868418622 0.500671444989
322.0 22.1828804391 0.500112411298
338.0 23.034523793 0.500535203198
355.0 23.9246189386 0.50027180984
373.0 24.8519445801 0.499348808141
392.0 25.8472006941 0.500717423348
411.0 26.8105552248 0.499897243002
432.0 27.8879493237 0.500908186243
453.0 28.9231265259 0.499062629217
476.0 30.0756593333 0.499458184509
500.0 31.2646484375 0.499555119372
525.0 32.5069856644 0.500716463752
551.0 33.770303579 0.500489566463
579.0 35.1159506929 0.499987032843
608.0 36.5098074567 0.500297876964
638.0 37.938720319 0.500491735055
670.0 39.4491943359 0.500575268008
703.0 40.9770285797 0.499537251699
739.0 42.6535827745 0.499891660794
776.0 44.3624085045 0.500188253369
814.0 46.0829485512 0.499260643107
855.0 47.9700933248 0.500839822171
898.0 49.8847973435 0.499717528144
943.0 51.8926269531 0.499639288912
990.0 53.9804517622 0.499898591976
1040.0 56.200680542 0.50093594005
1092.0 58.4671757519 0.500741899931
1146.0 60.8064071505 0.500691067536
1204.0 63.2933041505 0.500327513211
1264.0 65.8610274124 0.500584053758
1327.0 68.4990657863 0.499231190457
1393.0 71.2963913725 0.500080030078
1463.0 74.2254139217 0.500276195495
1536.0 77.2537293005 0.500305313224
1613.0 80.4046382285 0.499583850586
1694.0 83.727043372 0.499990573904
1778.0 87.1482685815 0.500384559177
1867.0 90.7234128672 0.499966896796
1961.0 94.4623285681 0.499218563976
2059.0 98.3696062173 0.499620695876
2162.0 102.459125061 0.500349999669
2270.0 106.692528534 0.50033202832
2383.0 111.103384304 0.500640459639
2503.0 115.686590344 0.499080363311
2628.0 120.502203226 0.499499559419
2759.0 125.512271729 0.49983572273
2897.0 130.719300766 0.499298482674
3042.0 136.169412296 0.49913706664
3194.0 141.867721704 0.499501154776
3354.0 147.831173401 0.49996049273
3522.0 154.060492916 0.500593760357
3698.0 160.468230848 0.499655063575
3883.0 167.212234375 0.499775653363
4077.0 174.251686707 0.500140228779
4281.0 181.591603765 0.500219092886
4495.0 189.251804733 0.500467001306
4720.0 197.189731635 0.499589372079
4956.0 205.484097576 0.499085012247
5203.0 214.234965706 0.500593440165
5464.0 223.307551384 0.500262052929
5737.0 232.761352152 0.500287352734
6024.0 242.615016768 0.499972098798
6325.0 252.921782509 0.500168003536
6641.0 263.708789051 0.500799815092
6973.0 274.836585302 0.499641791254
7322.0 286.65511013 0.500919984247
7688.0 298.822136898 0.50031622208
8073.0 311.497898655 0.499191097742
8476.0 324.770574751 0.499040779297
8900.0 338.698619898 0.499400337042
9345.0 353.28366394 0.500310349403
9812.0 368.33328125 0.499511182665
10303.0 384.128150178 0.499361709903
10818.0 400.721971185 0.500376056538
11359.0 417.989112705 0.500786529493
11927.0 435.928064575 0.500438429241
12523.0 454.670305805 0.500327979848
13150.0 474.131430626 0.499056286422
13807.0 494.591205201 0.499260914847
14498.0 516.031108398 0.499784591246
15222.0 538.324341203 0.499970241227
15984.0 561.51331602 0.499148255222
16783.0 585.913020676 0.499841854969
17622.0 611.263125 0.499674223614
18503.0 637.798856087 0.499911061154
19428.0 665.378045824 0.499361844188
20400.0 694.414306641 0.500078877676
21420.0 724.5984375 0.500099198154
22491.0 756.148331801 0.50027251562
23615.0 789.035610962 0.500204198252
24796.0 823.3046875 0.499633332713
26036.0 859.395441103 0.500542716723
27338.0 896.718337973 0.499695089846
28705.0 936.072695541 0.500562223666
30140.0 976.868616342 0.500111756373
31647.0 1019.65189178 0.500364531721
33229.0 1064.36640625 0.500756915881
34891.0 1111.05603457 0.500976489893
36635.0 1159.72308521 0.500919677629
38467.0 1210.4872757 0.500524397932
40390.0 1263.59382524 0.500467459326
42410.0 1318.6859375 0.49905640597
44530.0 1376.7748791 0.499493749089
46757.0 1437.69812345 0.500569273888
49095.0 1501.01830131 0.500670675897
51550.0 1566.63671875 0.499237562468
54127.0 1636.29012629 0.500923680567
56834.0 1708.07585949 0.499866578791
59675.0 1783.55534493 0.50015170856
62659.0 1862.63667969 0.500903047365
65792.0 1944.205 0.499145938059
69082.0 2030.63300781 0.50014814327
72536.0 2120.82796875 0.500901728019
76163.0 2214.2309668 0.499845083687
79971.0 2312.44268555 0.500079250866
83969.0 2414.92875977 0.500053292233
88168.0 2521.91476562 0.499772162426
92576.0 2634.438125 0.500847346883
97205.0 2750.97744141 0.500044259976
102065.0 2873.27386682 0.50013494447
107169.0 3001.56925781 0.500966962155
112527.0 3134.05277344 0.499359277995
118153.0 3273.88656004 0.499841728058
124061.0 3419.86954678 0.500056694118
130264.0 3572.083125 0.499786922985
136778.0 3732.00910156 0.500606412772
143616.0 3898.05015381 0.500043803347
150797.0 4072.33412203 0.500369474856
158337.0 4253.28344913 0.499261210599
166254.0 4443.72263672 0.499662600368
174567.0 4642.6145725 0.499868219428
183295.0 4849.66738939 0.499204319731
192460.0 5067.1109375 0.49956492566
202083.0 5295.42619718 0.5009029489
212187.0 5531.17351713 0.499495752278
222796.0 5778.77125 0.499175082347
233936.0 6038.68315903 0.499781559045
245633.0 6310.11559643 0.500141897281
257915.0 6593.95966797 0.500558104104
270811.0 6889.28374023 0.49988849054
284351.0 7199.02312012 0.499984918153
298569.0 7522.53925781 0.499850491289
313497.0 7861.91695312 0.500512633141
329172.0 8216.44171875 0.500955431158
345631.0 8584.21560584 0.499580908809
362912.0 8970.86321014 0.49954528047
381058.0 9377.59921875 0.500928586439
400111.0 9799.59363281 0.500457419154
420116.0 10240.3275 0.499749232259
441122.0 10702.8086816 0.499903349202
463178.0 11185.9296289 0.499833893425
486337.0 11692.985293 0.50065889582
510654.0 12217.7958984 0.499013733523
536187.0 12771.0946582 0.499396565094
562996.0 13349.1629687 0.49956325992
591146.0 13953.1238477 0.499511938535
620703.0 14584.0958789 0.499238757459
651739.0 15247.034238 0.500123108562
684326.0 15938.6475586 0.500397628866
718542.0 16658.3858203 0.499462563525
754469.0 17417.6241797 0.500715873657
792193.0 18203.4192285 0.499332497297
831802.0 19032.3445898 0.500171797909
873392.0 19898.6673437 0.500811190653
917062.0 20799.5751465 0.499995235796
962915.0 21745.5169678 0.500214883803
	\end{filecontents*}


\begin{filecontents*}[overwrite]{w_CSHF.dat}
	a	b
100.0 20.4000124708 0.500426997714
105.0 21.175205797 0.499509041874
110.0 21.9670349121 0.500300924286
115.0 22.7472959459 0.500675488435
121.0 23.6595748138 0.499997941564
127.0 24.5856311035 0.500775138535
134.0 25.6353101677 0.500273262204
140.0 26.5351593172 0.500336605278
147.0 27.5790605121 0.500542313073
155.0 28.7420268097 0.499615291663
162.0 29.7702560974 0.499916337447
171.0 31.0672787476 0.499507244296
179.0 32.217237253 0.499511319134
188.0 33.5152524623 0.500247313969
197.0 34.7694659424 0.4992677338
207.0 36.1844682813 0.499799810192
218.0 37.7182004058 0.499875899735
229.0 39.2246711731 0.49927842147
240.0 40.7409445882 0.499648638209
252.0 42.3974566269 0.50065329319
265.0 44.1167419434 0.499164880439
278.0 45.8740258694 0.49980238203
292.0 47.7146436316 0.499003075873
307.0 49.7026136112 0.499353511114
322.0 51.6606784524 0.499141373403
338.0 53.7398954296 0.499136588665
355.0 55.9828973055 0.500817398267
373.0 58.2891882369 0.500813194276
392.0 60.6784945608 0.499959685907
411.0 63.0563674539 0.499343398687
432.0 65.6873897311 0.499350864148
453.0 68.2947240399 0.499219806686
476.0 71.1741507387 0.500300796437
500.0 74.0835380415 0.499385595439
525.0 77.165210098 0.50037062289
551.0 80.3032195008 0.500212563981
579.0 83.6315123749 0.499394192741
608.0 87.0839440918 0.49929349899
638.0 90.6522784587 0.499678896944
670.0 94.4392921092 0.500209347201
703.0 98.2679740191 0.499660949516
739.0 102.464227775 0.500083578561
776.0 106.747110739 0.500478122487
814.0 111.071080017 0.499991495432
855.0 115.67851211 0.499010078114
898.0 120.613121698 0.500448242333
943.0 125.638173828 0.500046908335
990.0 130.871784338 0.499985480202
1040.0 136.414204979 0.500106149814
1092.0 142.106040221 0.499688210638
1146.0 148.08833997 0.500922973419
1204.0 154.260466208 0.499097499072
1264.0 160.774497206 0.499717533662
1327.0 167.4974389 0.499334954541
1393.0 174.597155493 0.500276713077
1463.0 182.009665256 0.500324775952
1536.0 189.655788574 0.499941938764
1613.0 197.792352905 0.500991756862
1694.0 206.060520257 0.499333818283
1778.0 214.723668126 0.499283623967
1867.0 223.883093813 0.499621654635
1961.0 233.563893554 0.500623411471
2059.0 243.43202827 0.499995624283
2162.0 253.718867788 0.499177869251
2270.0 264.686476403 0.500628319012
2383.0 275.858022236 0.499966067785
2503.0 287.72494087 0.499944599748
2628.0 299.923744003 0.499196481504
2759.0 312.810714093 0.499875006053
2897.0 326.150042696 0.499345176151
3042.0 340.261290413 0.500131431927
3194.0 354.739863281 0.49928182523
3354.0 370.135421619 0.500114969892
3522.0 385.963056557 0.499316125544
3698.0 402.703409202 0.500155034753
3883.0 420.112027279 0.50043618878
4077.0 438.10205434 0.499754757871
4281.0 457.055987642 0.499908624619
4495.0 476.762480134 0.499703643957
4720.0 497.50674948 0.500248854943
4956.0 519.133740234 0.500725955042
5203.0 541.337548564 0.499686636239
5464.0 564.806567831 0.499316615017
5737.0 589.610547551 0.50074087651
6024.0 614.874035515 0.499182572622
6325.0 641.853288337 0.500328905368
6641.0 669.89539629 0.500917714644
6973.0 698.590615983 0.499101480701
7322.0 729.219739532 0.499709854471
7688.0 761.06094101 0.499904759099
8073.0 794.528261719 0.500626039283
8476.0 829.128428421 0.500489929436
8900.0 865.359863281 0.50042350727
9345.0 902.877448915 0.499390858979
9812.0 942.392773437 0.499444279084
10303.0 983.920638853 0.500072016633
10818.0 1026.96803843 0.499881766257
11359.0 1072.18981968 0.500329539391
11927.0 1119.49825305 0.500954296866
12523.0 1168.3426465 0.500201841528
13150.0 1219.77274204 0.500210568757
13807.0 1273.47250857 0.500409146025
14498.0 1329.38448242 0.499972641785
15222.0 1387.81574321 0.499858750874
15984.0 1448.96602452 0.499664836036
16783.0 1513.3791626 0.500697572275
17622.0 1579.83230108 0.500069706331
18503.0 1650.09020277 0.500978872692
19428.0 1722.1896263 0.499542915498
20400.0 1798.48520279 0.499695368366
21420.0 1878.62732499 0.500613720469
22491.0 1961.2224455 0.499690390968
23615.0 2048.22361755 0.499919552864
24796.0 2138.47338867 0.499082121162
26036.0 2234.43907201 0.500611813505
27338.0 2333.1248499 0.499926916867
28705.0 2436.10560913 0.499082980348
30140.0 2545.57015811 0.500615083635
31647.0 2657.79556824 0.499455363672
33229.0 2776.0330835 0.499470915495
34891.0 2900.05262828 0.499913793415
36635.0 3029.13966897 0.499863242612
38467.0 3163.97415542 0.499672206545
40390.0 3306.30197145 0.500926624896
42410.0 3453.33011099 0.500428737327
44530.0 3606.72887573 0.499762852989
46757.0 3768.78428772 0.500575443229
49095.0 3937.03890945 0.500422176337
51550.0 4113.69153078 0.500909016519
54127.0 4296.00425246 0.499586888885
56834.0 4488.57502212 0.499758488873
59675.0 4689.55367937 0.499763703125
62659.0 4899.48636104 0.499613661645
65792.0 5118.44613281 0.499148232957
69082.0 5349.98985792 0.500392246002
72536.0 5589.08910219 0.499786172635
76163.0 5841.60359055 0.500714028702
79971.0 6104.20732086 0.500834562854
83969.0 6376.52406403 0.499745278556
88168.0 6665.20940002 0.500765262812
92576.0 6962.22062744 0.499342667173
97205.0 7277.0704216 0.500064057727
102065.0 7604.69495261 0.500046278073
107169.0 7948.27528588 0.50046925827
112527.0 8305.80732662 0.500177935969
118153.0 8677.73533028 0.499106117356
124061.0 9069.17788467 0.499079758682
130264.0 9480.8075589 0.499965287656
136778.0 9908.1193454 0.49965758518
143616.0 10354.3342676 0.499190269023
150797.0 10824.1676335 0.499824882469
158337.0 11315.0940793 0.500321835031
166254.0 11823.9900103 0.499400095636
174567.0 12364.0282158 0.500895420565
183295.0 12919.506234 0.499665327597
192460.0 13509.0703506 0.500893999988
202083.0 14115.3985683 0.499349489456
212187.0 14758.9452934 0.500303194846
222796.0 15426.1013488 0.499781630864
233936.0 16123.1149927 0.499107943008
245633.0 16857.2655126 0.499648311275
257915.0 17627.720472 0.500753672943
270811.0 18423.139251 0.499644826363
284351.0 19260.9202989 0.499777992405
298569.0 20136.49124 0.499773114977
313497.0 21051.4047513 0.499626189132
329172.0 22015.5498463 0.500775921611
345631.0 23015.0554549 0.500354133551
362912.0 24059.408197 0.499785110087
381058.0 25150.7234305 0.499068968585
400111.0 26300.7725444 0.499688657533
420116.0 27502.906825 0.500180241618
441122.0 28748.7798159 0.499032708339
463178.0 30073.087551 0.500774820083
486337.0 31434.2112215 0.499338956755
510654.0 32881.2113913 0.500838751439
536187.0 34371.9022872 0.49955565569
562996.0 35952.997924 0.500812761939
591146.0 37591.8745561 0.500371584381
620703.0 39304.7239491 0.499789331466
651739.0 41110.8669269 0.500674218964
684326.0 42982.5985778 0.499817915056
718542.0 44956.2501079 0.500449905609
754469.0 47001.4005967 0.499313054096
792193.0 49158.0606213 0.499686820502
831802.0 51412.8069942 0.499934937565
873392.0 53770.1690477 0.500056595447
917062.0 56234.7663072 0.500050220367
962915.0 58811.3659052 0.499913920418
\end{filecontents*}
		\begin{tikzpicture}
	\begin{loglogaxis}[width=0.4\textwidth, xlabel = $M$, ylabel = $w_\text{max}$, legend pos = outer north east]

	\addplot[smooth, blue] table[x=a,y=b] {w_SHF.dat};
\addlegendentry{SHF}

	\addplot[smooth, red] table[x=a,y=b] {w_CSHF.dat};
\addlegendentry{CSHF}
	\end{loglogaxis}
	\end{tikzpicture}

%	\caption{Value $w_{max}$ for which SHF and CSHF reach an error of 0.5 ($\pm 0.001$), for several values of $M$.}\label{fig:w_opt}
%\end{figure}

\section{Non-windowed DDFs in a wDDP setting}
\subsection{Lower Bound on the Saturation Resistance}
As said in the introduction, it has been proven \cite{10.1145/3297280.3297335} that all filters will reach saturation on the DDP setting. However, they sometimes prove to be efficient in some specific wDDP settings.
This bound is useful for several reasons, notably it provides an estimation of how close to optimality existing filters are.

\begin{restatable}{thm}{asymptoticSaturation}\label{thm:asymptoticSaturation}
	Let $E$ be a stream of $n$ elements uniformly selected from an alphabet of size $|\Gamma|$. For any DDF using $M$ bits of memory, the error probability $E_n = \FP_n + \FN_n$ satisfies
$
	E_n \geq 1 - \frac{1 - \left(1 - \frac{1}{|\Gamma|}\right)^{M}}{1 - \left(1 - \frac 1{|\Gamma|}\right)^n} 
$
	for any $n > M$.
\end{restatable}
In particular, the asymptotic error rate $E_\infty$ satisfies 
$
	E_\infty \geq \left(1 - \frac{1}{|\Gamma|}\right)^{M} \approx 1 - M/|\Gamma|.
$

%\begin{proof}
%	By definition, a perfect filter has the lowest possible error rate.
%	With $M$ bits of memory, a perfect filter can store at most $M$ elements in memory \cite[Theorem 2.1]{10.1145/3297280.3297335}. Up to reordering the stream, without loss of generality because it is random, we may assumethat the filter stores the $M$ last elements of the stream: any other strategy cannot yield a strictly lower error rate.
%	
%	If an element is already stored in the filter, then the optimal filter will necessarily answer \DUPLICATE. On the other hand, if the element is not in memory, a perfect filter can choose to answer randomly. Let $p$ be the probability that a filter answers \DUPLICATE when an element is not in memory. An optimal filter will lower the error rate of any filter using the same strategy with a different probability.
%	
%	An unseen element, by definition, will be unseen in the $M$ last elements of the stream, and hence will not be in the filter's memory, so the filter will return \UNSEEN with probability $1-p$. For this reason, this filter has an FP probability of $p$.
%	
%	On the other hand, $e^\star \dup E$ is classified as \UNSEEN if and only if it was not seen in the last $M$ elements of the stream, and the filter answers \UNSEEN. Let $D$ be the event \enquote{\emph{There is at least one duplicate in the stream}} and $C$ be the event \enquote{\emph{There is a duplicate of $e^\star$ in the $M$ previous elements of the stream}}. Then $e^\star$ triggers a false negative with probability
%	\begin{align*}
%		\FN_n &= (1 - \Pr[C | D])(1-p) =  \left(1 - \frac{\Pr[C \cap D]}{\Pr[D]}\right)(1-p)\\
%		& = \left(1 - \frac{\Pr[C]}{\Pr[D]}\right)(1-p) =  \left(1 - \frac{1 - \Pr[\bar C]}{1 - \Pr[\bar D]}\right)(1-p) \\
%		\FN_n &=  \left(1 - \frac{1 - \left(1 - \frac 1{|\Gamma|}\right)^M}{1 - \left(1 - \frac 1{|\Gamma|}\right)^n}\right)(1-p)
%	\end{align*}
%	Hence, the error probability of the filter is
%	\begin{align*}
%		E_n 
%		& = \FN_n + p
%		=  \left(1 - \frac{1 - \left(1 - \frac{1}{|\Gamma|}\right)^{M}}{1 - \left(1 - \frac 1{|\Gamma|}\right)^n}\right)(1-p) + p \\
%		& = 1 - \frac{1-\left(1 - \frac{1}{|\Gamma|}\right)^{M}}{1 - \left(1 - \frac 1{|\Gamma|}\right)^n}(1-p),
%	\end{align*}
%	which is minimized when $p = 0$. 
%\end{proof}

Note, as highlighted in the proof, that this bound is \emph{not tight}: better bounds may exist, the study of which we leave as an open question for future work.


\subsection{Saturation Resistance of DDFs}\label{sub:saturation}
We now evaluate the saturation rate for several DDFs, in the original DDP setting (without sliding window). Parameters are chosen to yield equivalent memory footprints and were taken from \cite{10.1145/3297280.3297335}, namely:
%
\begin{itemize}
	\item QHT \cite{10.1145/3297280.3297335}, 1 bucket per row, 3 bits per fingerprint.
	\item SQF \cite{Dut13}, 1 bucket per row, $r = 2$ and $r' = 1$
	\item Cuckoo Filter \cite{Fan14}, cells containing 1 element of 3 bits each
	\item Stable Bloom Filter (SBF) \cite{Den06}, 2 bits per cell, 2 hash functions, targeted FPR of 0.02
%	\item $A2$ Filter \cite{Yoo10}, targeted FPR of $0.1$ on the sliding window.
%	\item Block-Decaying Bloom Filter (b\_DBF) \cite{She08}, sliding window of 6000 elements.
\end{itemize}
%


These filters are run against a stream of uniformly sampled elements from an alphabet of $2^{26}$ elements. This results in around 8\% duplicates amongst the 150 000 000 elements in the longest stream used. %As a comparison, we also run the same benchmark against a stream obtained from crawling data, the same one as \cite{10.1145/3297280.3297335}, a real-world source which is highly non-uniform --- note that in this case, our lower bound on the error rate does not apply, owing to the stream being compressible. In that stream, around 10\% were duplicates amongst the 150 000 000 elements.
Results are plotted in Figure~\ref{fig:graph_n}.

The best results are given by the following filters, in order: QHT, SQF, Cuckoo and SBF. We also observe that QHT and SQF have error rates relatively close to the lower bound, hence suggesting that these filters are close to optimality, especially since the lower bound is not tight.


\subsection{Performance in wDDP}
We now consider the performance of the filters just discussed in the \emph{windowed} setting, for which they were \emph{not} designed. In particular, it is not possible to adjust their parameters as a function of $w$.

Remarkably, some of these filters still outperform dedicated windowed filters for some window sizes at least, as shown in Figure~\ref{fig:nwddf-in-wddp}. In this benchmark, we used the following filters:
\begin{itemize}
	\item block decaying Bloom Filter\footnote{Note that by design, a b\_DBF of $10^5$ bits cannot be built over a sliding window bigger than $6000$.} (b\_DBF) \cite{She08}, sliding window of size $w$
	\item A2 filter \cite{Yoo10}, changing subfilter every $w/2$ insertions
	\item QHT \cite{10.1145/1060745.1060753}, 1 bucket per row, $3$ bits per fingerprint
\end{itemize}

Nevertheless, we will now discuss the queuing construction, which allow us to build windowed filters from the DDP filters.  

\begin{figure}[t]
\makebox[\textwidth]{\makebox[1.25\textwidth]{%
	\begin{minipage}{.5\textwidth}
	\centering
		\begin{filecontents*}[overwrite]{b_qht.dat}
w	 FPR 	 FNR 	 Error 	 1-E 	 Phi 	 percentDup
100 	24.29 	0 	24.29 	75.71 	3.57 	0.04
111 	24.28 	0 	24.28 	75.72 	3.75 	0.05
123 	24.28 	0 	24.28 	75.72 	3.95 	0.05
138 	24.27 	0 	24.27 	75.73 	4.19 	0.06
153 	24.27 	0 	24.27 	75.73 	4.38 	0.06
171 	24.27 	0 	24.27 	75.73 	4.61 	0.07
190 	24.26 	0 	24.26 	75.74 	4.84 	0.08
212 	24.25 	0 	24.25 	75.75 	5.08 	0.08
236 	24.25 	0.11 	24.36 	75.64 	5.36 	0.09
262 	24.24 	0.29 	24.53 	75.47 	5.67 	0.1
292 	24.23 	0.43 	24.66 	75.34 	5.97 	0.12
325 	24.22 	0.39 	24.61 	75.39 	6.27 	0.13
362 	24.21 	0.42 	24.63 	75.37 	6.62 	0.14
404 	24.2 	0.38 	24.57 	75.43 	7.02 	0.16
449 	24.18 	0.33 	24.52 	75.48 	7.46 	0.18
500 	24.17 	0.35 	24.52 	75.48 	7.86 	0.2
557 	24.15 	0.62 	24.77 	75.23 	8.29 	0.22
621 	24.13 	0.61 	24.74 	75.26 	8.69 	0.25
691 	24.11 	0.7 	24.82 	75.18 	9.1 	0.27
769 	24.09 	0.84 	24.93 	75.07 	9.54 	0.3
857 	24.07 	0.97 	25.04 	74.96 	10.03 	0.33
954 	24.04 	0.95 	25 	75 	10.57 	0.37
1062 	24.02 	1.29 	25.31 	74.69 	11.03 	0.4
1183 	23.98 	1.42 	25.41 	74.59 	11.63 	0.45
1317 	23.95 	1.55 	25.5 	74.5 	12.21 	0.5
1466 	23.9 	1.76 	25.66 	74.34 	12.89 	0.56
1633 	23.86 	2 	25.85 	74.15 	13.58 	0.62
1818 	23.8 	2.19 	26 	74 	14.31 	0.69
2024 	23.75 	2.43 	26.17 	73.83 	15.05 	0.77
2254 	23.69 	2.76 	26.45 	73.55 	15.82 	0.86
2509 	23.62 	3.1 	26.72 	73.28 	16.55 	0.95
2794 	23.54 	3.43 	26.97 	73.03 	17.43 	1.06
3111 	23.46 	3.83 	27.29 	72.71 	18.31 	1.18
3464 	23.36 	4.43 	27.79 	72.21 	19.21 	1.32
3857 	23.26 	4.94 	28.21 	71.79 	20.14 	1.47
4294 	23.15 	5.48 	28.63 	71.37 	21.09 	1.63
4781 	23.03 	6.1 	29.13 	70.87 	22.05 	1.81
5324 	22.89 	6.63 	29.52 	70.48 	23.12 	2.02
5928 	22.74 	7.31 	30.05 	69.95 	24.18 	2.25
6600 	22.59 	8.15 	30.73 	69.27 	25.21 	2.5
7348 	22.41 	8.94 	31.36 	68.64 	26.27 	2.77
8182 	22.23 	9.88 	32.11 	67.89 	27.35 	3.08
9110 	22.02 	10.74 	32.75 	67.25 	28.49 	3.42
10143 	21.8 	11.77 	33.56 	66.44 	29.58 	3.79
11294 	21.56 	12.96 	34.52 	65.48 	30.64 	4.21
12574 	21.3 	14.15 	35.45 	64.55 	31.76 	4.67
14001 	21.03 	15.49 	36.52 	63.48 	32.78 	5.18
15588 	20.74 	16.93 	37.66 	62.34 	33.82 	5.75
17356 	20.43 	18.49 	38.91 	61.09 	34.77 	6.37
19325 	20.11 	20.19 	40.3 	59.7 	35.61 	7.05
21517 	19.77 	21.87 	41.64 	58.36 	36.47 	7.8
23957 	19.42 	23.74 	43.16 	56.84 	37.19 	8.62
26674 	19.05 	25.66 	44.7 	55.3 	37.85 	9.53
29699 	18.67 	27.78 	46.45 	53.55 	38.33 	10.54
33067 	18.28 	29.94 	48.22 	51.78 	38.73 	11.66
36818 	17.89 	32.14 	50.02 	49.98 	39.01 	12.87
40993 	17.5 	34.46 	51.97 	48.03 	39.07 	14.19
45642 	17.12 	36.82 	53.93 	46.07 	39 	15.63
50819 	16.74 	39.23 	55.97 	44.03 	38.73 	17.2
56582 	16.37 	41.6 	57.98 	42.02 	38.35 	18.9
63000 	16.04 	43.97 	60 	40 	37.78 	20.71
64472 	15.97 	44.5 	60.47 	39.53 	37.61 	21.12
65979 	15.9 	44.99 	60.89 	39.11 	37.47 	21.52
67521 	15.82 	45.45 	61.27 	38.73 	37.36 	21.94
69099 	15.75 	45.92 	61.67 	38.33 	37.23 	22.36
70714 	15.67 	46.45 	62.12 	37.88 	37.05 	22.82
72366 	15.61 	46.95 	62.56 	37.44 	36.88 	23.27
74057 	15.54 	47.45 	62.99 	37.01 	36.7 	23.73
75788 	15.47 	47.95 	63.42 	36.58 	36.51 	24.19
77559 	15.41 	48.48 	63.89 	36.11 	36.29 	24.66
79372 	15.35 	48.96 	64.31 	35.69 	36.09 	25.13
81227 	15.28 	49.43 	64.71 	35.29 	35.91 	25.61
83125 	15.23 	49.91 	65.14 	34.86 	35.68 	26.08
85068 	15.17 	50.39 	65.56 	34.44 	35.46 	26.57
87056 	15.1 	50.84 	65.94 	34.06 	35.27 	27.07
89091 	15.04 	51.31 	66.35 	33.65 	35.05 	27.57
91173 	14.98 	51.77 	66.75 	33.25 	34.83 	28.08
93304 	14.93 	52.25 	67.18 	32.82 	34.58 	28.6
95484 	14.88 	52.72 	67.6 	32.4 	34.32 	29.13
97716 	14.82 	53.14 	67.96 	32.04 	34.11 	29.65
	\end{filecontents*}


\begin{filecontents*}[overwrite]{b_bdbf.dat}
w	 FPR 	 FNR 	 Error 	 1-E 	 Phi 	 percentDup
100 	0.02 	0 	0.02 	99.98 	80.54 	0.04
111 	0.04 	0 	0.04 	99.96 	73.96 	0.05
123 	0.03 	0 	0.03 	99.97 	81.42 	0.05
138 	0.03 	0 	0.03 	99.97 	83 	0.06
153 	0.03 	0 	0.03 	99.97 	84.31 	0.06
171 	0.02 	0 	0.02 	99.98 	87.19 	0.07
190 	0.03 	0 	0.03 	99.97 	83.49 	0.08
212 	0.02 	0 	0.02 	99.98 	90.24 	0.08
236 	0.02 	0 	0.02 	99.98 	89.88 	0.09
262 	0.02 	0 	0.02 	99.98 	91.24 	0.1
292 	0.03 	0 	0.03 	99.97 	89.11 	0.12
325 	0.03 	0 	0.03 	99.97 	88.9 	0.13
362 	0.03 	0 	0.03 	99.97 	89.7 	0.14
404 	0.05 	0 	0.05 	99.95 	87.02 	0.16
449 	0.08 	0 	0.08 	99.92 	83.34 	0.18
500 	0.09 	0 	0.09 	99.91 	82.49 	0.2
557 	0.14 	0 	0.14 	99.86 	78.17 	0.22
621 	0.21 	0 	0.21 	99.79 	73.44 	0.25
691 	0.32 	0 	0.32 	99.68 	67.68 	0.27
769 	0.42 	0 	0.42 	99.58 	64.4 	0.3
857 	0.6 	0 	0.6 	99.4 	59.5 	0.33
954 	0.78 	0 	0.78 	99.22 	56.38 	0.37
1062 	1.11 	0 	1.11 	98.89 	51.4 	0.4
1183 	1.46 	0 	1.46 	98.54 	48.26 	0.45
1317 	1.93 	0 	1.93 	98.07 	44.85 	0.5
1466 	2.58 	0 	2.58 	97.42 	41.66 	0.56
1633 	3.14 	0 	3.14 	96.86 	40.1 	0.62
1818 	4.12 	0 	4.12 	95.88 	37.25 	0.69
2024 	5.18 	0 	5.18 	94.82 	35.16 	0.77
2254 	6.18 	0 	6.18 	93.82 	33.97 	0.86
2509 	7.61 	0 	7.61 	92.39 	32.13 	0.95
2794 	9.64 	0 	9.64 	90.36 	30.06 	1.06
3111 	11.05 	0 	11.05 	88.95 	29.46 	1.18
3464 	12.92 	0 	12.92 	87.08 	28.59 	1.32
3857 	15.52 	0 	15.52 	84.48 	27.22 	1.47
4294 	18.72 	0 	18.72 	81.28 	25.73 	1.63
4781 	24.15 	0 	24.15 	75.85 	23.22 	1.81
5324 	23.5 	0 	23.5 	76.5 	24.85 	2.02
5928 	26.63 	0 	26.63 	73.37 	24.16 	2.25
6600 	30.42 	0 	30.42 	69.58 	23.26 	2.5
7348 	34.86 	0 	34.86 	65.14 	22.19 	2.77
8182 	38.97 	0 	38.97 	61.03 	21.46 	3.08
9110 	42.05 	0 	42.05 	57.95 	21.22 	3.42
10143 	55.34 	0 	55.34 	44.66 	17.24 	3.79
11294 	49.14 	0 	49.14 	50.86 	20.42 	4.21
12574 	53.16 	0 	53.16 	46.84 	19.89 	4.67
14001 	68.95 	0 	68.95 	31.05 	15.09 	5.18
15588 	73.58 	0 	73.58 	26.42 	14.22 	5.75
17356 	82.08 	0 	82.08 	17.92 	11.71 	6.37
19325 	84.4 	0 	84.4 	15.6 	11.34 	7.05
21517 	86.41 	0 	86.41 	13.59 	11 	7.8
23957 	88.39 	0 	88.39 	11.61 	10.58 	8.62
26674 	89.99 	0 	89.99 	10.01 	10.24 	9.53
29699 	91.44 	0 	91.44 	8.56 	9.88 	10.54
33067 	92.59 	0 	92.59 	7.41 	9.62 	11.66
36818 	93.47 	0 	93.47 	6.53 	9.44 	12.87
40993 	94.21 	0 	94.21 	5.79 	9.3 	14.19
45642 	94.78 	0 	94.78 	5.22 	9.24 	15.63
50819 	95.04 	0 	95.04 	4.96 	9.44 	17.2
56582 	95.25 	0 	95.25 	4.75 	9.66 	18.9
\end{filecontents*}

\begin{filecontents*}[overwrite]{b_a2.dat}
w	 FPR 	 FNR 	 Error 	 1-E 	 Phi 	 percentDup
100 	43.91 	0 	43.91 	56.09 	2.29 	0.04
111 	43.91 	0 	43.91 	56.09 	2.4 	0.05
123 	43.91 	0 	43.91 	56.09 	2.53 	0.05
138 	43.91 	0 	43.91 	56.09 	2.69 	0.06
153 	43.9 	0 	43.9 	56.1 	2.8 	0.06
171 	43.9 	0 	43.9 	56.1 	2.95 	0.07
190 	43.89 	0 	43.89 	56.11 	3.1 	0.08
212 	43.89 	0 	43.89 	56.11 	3.26 	0.08
236 	43.89 	0 	43.89 	56.11 	3.44 	0.09
262 	43.88 	0 	43.88 	56.12 	3.64 	0.1
292 	43.87 	0 	43.87 	56.13 	3.85 	0.12
325 	43.87 	0 	43.87 	56.13 	4.04 	0.13
362 	43.86 	0 	43.86 	56.14 	4.26 	0.14
404 	43.85 	0 	43.85 	56.15 	4.51 	0.16
449 	43.84 	0 	43.84 	56.16 	4.8 	0.18
500 	43.82 	0 	43.82 	56.18 	5.06 	0.2
557 	43.81 	0 	43.81 	56.19 	5.35 	0.22
621 	43.8 	0 	43.8 	56.2 	5.61 	0.25
691 	43.79 	0 	43.79 	56.21 	5.88 	0.27
769 	43.77 	0 	43.77 	56.23 	6.17 	0.3
857 	43.75 	0 	43.75 	56.25 	6.51 	0.33
954 	43.73 	0 	43.73 	56.27 	6.85 	0.37
1062 	43.71 	0 	43.71 	56.29 	7.18 	0.4
1183 	43.68 	0 	43.68 	56.32 	7.59 	0.45
1317 	43.66 	0 	43.66 	56.34 	7.98 	0.5
1466 	43.62 	0 	43.62 	56.38 	8.45 	0.56
1633 	43.59 	0 	43.59 	56.41 	8.93 	0.62
1818 	43.55 	0 	43.55 	56.45 	9.43 	0.69
2024 	43.5 	0 	43.5 	56.5 	9.96 	0.77
2254 	43.45 	0 	43.45 	56.55 	10.51 	0.86
2509 	43.4 	0 	43.4 	56.6 	11.05 	0.95
2794 	43.34 	0 	43.34 	56.66 	11.69 	1.06
3111 	43.27 	0 	43.27 	56.73 	12.35 	1.18
3464 	43.19 	0 	43.19 	56.81 	13.07 	1.32
3857 	43.1 	0 	43.1 	56.9 	13.8 	1.47
4294 	43.01 	0 	43.01 	56.99 	14.56 	1.63
4781 	42.9 	0 	42.9 	57.1 	15.35 	1.81
5324 	42.78 	0 	42.78 	57.22 	16.22 	2.02
5928 	42.65 	0 	42.65 	57.35 	17.14 	2.25
6600 	42.5 	0 	42.5 	57.5 	18.09 	2.5
7348 	42.34 	0 	42.34 	57.66 	19.07 	2.77
8182 	42.15 	0 	42.15 	57.85 	20.14 	3.08
9110 	41.95 	0 	41.95 	58.05 	21.25 	3.42
10143 	41.73 	0 	41.73 	58.27 	22.43 	3.79
11294 	41.47 	0 	41.47 	58.53 	23.67 	4.21
12574 	41.19 	0 	41.19 	58.81 	25.01 	4.67
14001 	40.88 	0 	40.88 	59.12 	26.39 	5.18
15588 	40.52 	0 	40.52 	59.48 	27.89 	5.75
17356 	40.12 	0 	40.12 	59.88 	29.46 	6.37
19325 	39.69 	0 	39.69 	60.31 	31.1 	7.05
21517 	39.2 	0.09 	39.3 	60.7 	32.79 	7.8
23957 	38.72 	0.76 	39.48 	60.52 	34.22 	8.62
26674 	38.27 	2.25 	40.52 	59.48 	35.19 	9.53
29699 	37.86 	4.5 	42.37 	57.63 	35.66 	10.54
33067 	37.52 	7.47 	44.99 	55.01 	35.57 	11.66
36818 	37.32 	11.25 	48.57 	51.43 	34.7 	12.87
40993 	37.23 	15.49 	52.72 	47.28 	33.24 	14.19
45642 	37.22 	19.83 	57.05 	42.95 	31.43 	15.63
50819 	37.22 	23.71 	60.93 	39.07 	29.71 	17.2
56582 	37.19 	27.11 	64.3 	35.7 	28.17 	18.9
63000 	37.16 	30.1 	67.26 	32.74 	26.73 	20.71
64472 	37.16 	30.74 	67.9 	32.1 	26.4 	21.12
65979 	37.15 	31.33 	68.48 	31.52 	26.1 	21.52
67521 	37.15 	31.91 	69.06 	30.94 	25.8 	21.94
69099 	37.15 	32.49 	69.64 	30.36 	25.49 	22.36
70714 	37.14 	33.06 	70.19 	29.81 	25.2 	22.82
72366 	37.14 	33.64 	70.77 	29.23 	24.89 	23.27
74057 	37.13 	34.19 	71.32 	28.68 	24.58 	23.73
75788 	37.12 	34.7 	71.82 	28.18 	24.31 	24.19
77559 	37.12 	35.23 	72.35 	27.65 	24.01 	24.66
79372 	37.12 	35.75 	72.87 	27.13 	23.71 	25.13
81227 	37.11 	36.23 	73.35 	26.65 	23.44 	25.61
83125 	37.11 	36.7 	73.81 	26.19 	23.17 	26.08
85068 	37.09 	37.15 	74.24 	25.76 	22.92 	26.57
87056 	37.09 	37.6 	74.69 	25.31 	22.66 	27.07
89091 	37.08 	38.06 	75.14 	24.86 	22.39 	27.57
91173 	37.08 	38.51 	75.6 	24.4 	22.1 	28.08
93304 	37.08 	38.96 	76.04 	23.96 	21.82 	28.6
95484 	37.08 	39.37 	76.45 	23.55 	21.56 	29.13
97716 	37.07 	39.77 	76.84 	23.16 	21.31 	29.65
\end{filecontents*}

\begin{filecontents*}[overwrite]{b_a22.dat}
w	 FPR 	 FNR 	 Error 	 1-E 	 Phi 	 percentDup
100 	0 	29.27 	29.27 	70.73 	83.52 	0.04
112 	0 	27.07 	27.07 	72.93 	84.63 	0.05
125 	0 	28.09 	28.1 	71.9 	83.77 	0.05
141 	0 	27.51 	27.51 	72.49 	83.75 	0.06
158 	0 	24.28 	24.28 	75.72 	85.84 	0.06
177 	0 	26.6 	26.6 	73.4 	84.29 	0.07
199 	0 	23.23 	23.24 	76.76 	86.04 	0.08
223 	0 	24.19 	24.2 	75.8 	85.57 	0.09
251 	0 	24.37 	24.37 	75.63 	85.42 	0.1
281 	0 	25.11 	25.12 	74.88 	84.72 	0.11
316 	0.01 	25.7 	25.71 	74.29 	83.85 	0.12
354 	0.01 	24.19 	24.2 	75.8 	84.61 	0.14
398 	0.01 	24.7 	24.7 	75.3 	84.26 	0.16
446 	0.01 	25.93 	25.94 	74.06 	83.13 	0.18
501 	0.01 	26.36 	26.37 	73.63 	82.36 	0.2
562 	0.02 	25.82 	25.83 	74.17 	82.39 	0.23
630 	0.02 	24.39 	24.41 	75.59 	82.77 	0.25
707 	0.03 	24.54 	24.56 	75.44 	81.81 	0.28
794 	0.03 	23.38 	23.41 	76.59 	82.15 	0.31
891 	0.04 	22.75 	22.8 	77.2 	81.61 	0.34
999 	0.05 	23.74 	23.79 	76.21 	80.35 	0.38
1122 	0.07 	23.6 	23.67 	76.33 	79.57 	0.43
1258 	0.08 	23.66 	23.74 	76.26 	78.79 	0.47
1412 	0.1 	23.95 	24.05 	75.95 	77.94 	0.53
1584 	0.13 	24.66 	24.79 	75.21 	76.46 	0.6
1778 	0.16 	25.17 	25.33 	74.67 	75.15 	0.68
1995 	0.2 	24.64 	24.83 	75.17 	74.8 	0.76
2238 	0.26 	24.88 	25.13 	74.87 	73.12 	0.85
2511 	0.32 	24.84 	25.15 	74.85 	71.95 	0.95
2818 	0.4 	24.14 	24.54 	75.46 	71.18 	1.07
3162 	0.49 	25.1 	25.59 	74.41 	69.37 	1.2
3548 	0.61 	24.78 	25.39 	74.61 	68.27 	1.36
3981 	0.77 	24.41 	25.18 	74.82 	66.85 	1.51
4466 	0.97 	24.68 	25.66 	74.34 	64.98 	1.7
5011 	1.21 	24.14 	25.35 	74.65 	63.74 	1.91
5623 	1.5 	24.33 	25.83 	74.17 	62 	2.14
6309 	1.89 	23.69 	25.58 	74.42 	60.47 	2.39
7079 	2.33 	22.99 	25.33 	74.67 	59.18 	2.67
7943 	2.9 	22.73 	25.64 	74.36 	57.43 	2.99
8912 	3.59 	22.29 	25.88 	74.12 	55.81 	3.35
9999 	4.47 	21.76 	26.22 	73.78 	54.06 	3.74
11220 	5.51 	21.63 	27.14 	72.86 	52.14 	4.18
12589 	6.77 	20.99 	27.76 	72.24 	50.52 	4.68
14125 	8.34 	20.19 	28.53 	71.47 	48.81 	5.22
15848 	10.21 	19.43 	29.64 	70.36 	47.1 	5.84
17782 	12.48 	18.25 	30.73 	69.27 	45.52 	6.52
19952 	15.23 	16.84 	32.07 	67.93 	43.93 	7.26
22387 	18.51 	15.55 	34.06 	65.94 	42.2 	8.1
25118 	22.44 	14.21 	36.66 	63.34 	40.33 	9.01
28183 	27.06 	12.39 	39.44 	60.56 	38.66 	10.04
31622 	32.49 	10.66 	43.16 	56.84 	36.76 	11.19
35481 	37.38 	9.88 	47.26 	52.74 	35.06 	12.43
39810 	37.24 	14.31 	51.55 	48.45 	33.69 	13.82
44668 	37.22 	19.01 	56.24 	43.76 	31.78 	15.34
50118 	37.22 	23.26 	60.48 	39.52 	29.91 	17
56234 	37.19 	26.94 	64.13 	35.87 	28.24 	18.81
63095 	37.16 	30.14 	67.3 	32.7 	26.71 	20.73
70794 	37.14 	33.09 	70.22 	29.78 	25.19 	22.84
79432 	37.12 	35.77 	72.89 	27.11 	23.7 	25.15
89125 	37.08 	38.06 	75.14 	24.86 	22.38 	27.58
\end{filecontents*}
		\begin{tikzpicture}
	\begin{semilogxaxis}[xlabel = w, ylabel = Error, legend pos = outer north east]

	\addplot table[x=w,y=Error] {b_qht.dat};
\addlegendentry{QHT}

	\addplot table[x=w,y=Error] {b_bdbf.dat};
\addlegendentry{BDBF}

	\addplot table[x=w,y=Error] {b_a22.dat};
\addlegendentry{A2}
	\end{semilogxaxis}
	\end{tikzpicture}
	\caption{Error rates for QHT, b\_DBF, and A2. While A2 and b\_DBF were designed and adjusted to the wDDP, this is not the case of QHT. Still, QHT outperforms these filters for some values of $w$.}
	\label{fig:nwddf-in-wddp}
	\end{minipage}
	\hfill
	\begin{minipage}{.7\textwidth}
	\centering
	\begin{tikzpicture}[scale=0.4]
	
	% F_0 under construction
	\draw[pattern=north west lines, pattern color=gray, draw=none] (1, 0) rectangle (2, 1);
	\draw (0, 0) rectangle (2, 1) node[midway] {$\mathcal F_0$};

	
	% Other subfilters
	\draw[pattern=north west lines, pattern color=gray] (3, 0) rectangle (5, 1) node[midway] {$\mathcal F_1$};
	\draw[pattern=north west lines, pattern color=gray] (6, 0) rectangle (8, 1) node[midway] {$\mathcal F_2$};
	\node at (9.5, 0.5) {...};
	\draw[pattern=north west lines, pattern color=gray] (11, 0) rectangle (13, 1) node[midway] {$\mathcal F_{L-1}$};
	
	% old subfilters
	\draw[dashed] (14, 0) rectangle (16, 1) node[midway] {$\mathcal F_{L}$};
	\draw[dashed] (17, 0) rectangle (19, 1) node[midway] {$\mathcal F_{L+1}$};
	
	% arrows
	\draw [<->,>=latex] (6, 1.5) -- (8, 1.5) node[above, midway] {$c$ elements};
	\draw [<->,>=latex] (0, -3) -- (13, -3) node[below, midway] {Capacity of $Lc \approx w$ elements};
	\draw [<->,>=latex] (1, 3) -- (15, 3) node[above, midway] {Current sliding window (of size $w$)};
	
	%stream 
	\draw [<-, >=latex] (0, 5.5) -- (19, 5.5) node[above, midway] {Stream (most recent on the left)};
	\foreach \x in {1, 3,...,18}
	\draw (\x, 5.7) -- node[pos=0.5] (point\x) {} (\x, 5.3);
	
	\foreach \x in {2, 4,...,18}
	\draw (\x, 5.6) -- node[pos=0.5] (point\x) {} (\x, 5.4);

	\path (point1) node [below] {$e_m$};
	\foreach \x [evaluate=\x as \i using int(\x-1)] in {3, 5,..., 9}
	\path (point\x) node [below] {$e_{m-\i}$};
	\path (point11) node [below] {$\dots$};
	\path (point15) node [below] {$e_{m - w + 1}$};
	
	%queueing filter
	\draw (-0.5, 2.5) rectangle (13.5, -2.5) node[anchor=south east] {Queuing filter};
	
	% legend
	\draw[draw=none, pattern=north west lines, pattern color=gray] (0.5, -5) rectangle (1, -5.5);
	\draw (0, -5) rectangle (1, -5.5) node[midway] {\tiny $\mathcal F$};
	\draw (1.25, -5.25) node[anchor=west] {\tiny Subfilter being populated};
	
	\draw[pattern=north west lines, pattern color=gray] (7, -5) rectangle (8, -5.5) node[midway] {\tiny $\mathcal F_i$};
	\draw (8.25, -5.25) node[anchor=west] {\tiny Populated active subfilter};
	
	\draw[dashed] (14, -5) rectangle (15, -5.5) node[midway] {\tiny $\mathcal F_j$};
	\draw (1 5.25, -5.25) node[anchor=west] {\tiny Expired subfilter};
	\end{tikzpicture}
	\caption{Architecture of the queuing filter. The filter is composed of $L$ subfilters $\mathcal F$, each containing up to $c$ elements. Once the newest subfilter has inserted $c$ elements in its structure, the oldest one is expired. As such, it is dropped and a new one is created and put under population at the beginning of the queue. In this example, the sub-sliding window of $\mathcal F_1$ is $(e_{m-2}, e_{m-3}, e_{m-4})$.} \label{fig:scheme}
	\end{minipage}
}}
\end{figure}


\section{Queuing filters}\label{sec:queue}
We now describe the queuing construction, which produces a sliding window DDF from any DDF. We first give the description of the setup, before studying the theoretical error rates. 
A scheme describing our structure is detailed in Figure~\ref{fig:scheme}.

\subsection{The queuing construction}

\paragraph{Principle of operation.}
Let $\mathcal F$ be a DDF. Rather than allocating the whole memory to $\mathcal F$, we will create $L$ copies of $\mathcal F$, each using a fraction of the available memory. Each of these \emph{subfilters} as a limited timespan, and is allowed up to $c$ insertions. The subfilters are organised in a queue.

When inserting a new element in the queuing filter, it is inserted in the topmost subfilter of the queue. After $c$ insertions, a new empty filter is added to the queue, and the oldest subfilter is popped and erased.

As such, we can consider that each subfilter operates on a sub-sliding window of size $c$, which makes the overall construction a DDF operating over a sliding window of size $w = cL$.

\paragraph{Insertion and lookup.}
The filter returns \DUPLICATE if and only if at least one subfilter does. Insertion is a simple insertion in the topmost subfilter.

\paragraph{Queue update.}
After $c$ insertions, the last filter of the queue is dropped, and a new (empty) filter is appended in front of the queue.

\paragraph{Pseudocode.}
We give a brief pseudocode for the queuing filter's functions \textsf{Lookup} and \textsf{Insert}, as well as a \textsf{Setup} function for initialisation, in Algorithm~\ref{alg:queuing}. We introduced for simplicity a constructor $\mathcal F.\textsf{Setup}$ that takes as input an integer $M$ and outputs an initialized empty filter $\mathcal F$ of size at most $M$. Here \texttt{subfilters} is a FIFO that has a \texttt{pop} and \texttt{push\_first} operations, which respectevely removes the last element in the queue or inserts a new item in first position.\footnote{Two standard ways to implement this are a double-ended queue (deque) and a ring buffer.}   

\begin{algorithm}[]
	\caption{Queuing Filter Setup, Lookup and Insert}\label{alg:queuing}
	
	\begin{algorithmic}[1]
	\Function{Setup}{$\mathcal F, M, L, c$}
		\Comment $M$ is the available memory, $\mathcal F$ the subfilter structure, $L$ the number of subfilters and $c$ the number of insertions per subfilter
		\State subfilters $\gets \emptyset$
		\State counter $\gets 0$
		\State $m \gets \lfloor M/L \rfloor$
		\For{$i$ from $0$ to $L-1$}
			\State subfilters.push\_first$(\mathcal F.\mathsf{Setup}(m))$
		\EndFor
		\State \textbf{store} (subfilters, $L$, $m$, counter)
%		\State\Return filter
	\EndFunction 
	\end{algorithmic}

	\begin{multicols}{2}
	\begin{algorithmic}[1]
	\Function{Lookup}{$e$}
		\For{$i$ from $0$ to $L-1$}
			\If{subfilters[i]$.\mathsf{Lookup(e)}$}
				\State \Return $\DUPLICATE$
			\EndIf
		\EndFor
		\State \Return \UNSEEN
	\EndFunction
	\item[]
	\end{algorithmic}

	\begin{algorithmic}[1]
	\Function{Insert}{$e$}
		\State subfilters[0]$.\mathsf{Insert}(e)$
		\State counter $\gets$ counter + 1 
		\If{counter == $c$}
			\State subfilters.pop()
			\State subfilters.push\_first($\mathcal F.\textsf{Setup}(m)$)
		\EndIf
	\EndFunction
	\end{algorithmic}
	\end{multicols}
\end{algorithm}

\subsection{Error rate analysis}
The queuing filter's properties can be derived from the subfilters'. False positive and false negative rates are of particular interest. In this section we consider a queuing filter $\mathcal Q$ with $L$ subfilters of type $\mathcal F$ and 
capacity $c$ (which means that the last subfilter is dropped after $c$ insertions).

\paragraph{Remark.}
	\label{subsub:number_elements}
	By definition, after $c$ insertions the last subfilter is dropped.
Information-theoretically, this means that all the information related to the elements inserted in that subfilter has been lost, and there are $c$ such elements by design. 
Therefore, in the steady-state regime, the queuing filter holds information about at least $c(L-1)$  elements (immediately after deleting the last subfilter) and at most $cL$ elements (immediately before this deletion). 

As a result, if $w < cL$, the queuing filter can hold information about \emph{more than $w$ elements}.

\subsubsection{Probability of False Positive}

\begin{restatable}{thm}{pfpQueue}
	\label{thm:queue-pfp}
	 Let $\FP_{\mathcal Q, m}^w$ be the probability of false positive %(def. \ref{def:pfp-pfn})
	 of $\mathcal Q$ after $m > w$ insertions, over a sliding window of size $w = cL$, we have \\
	$$%\begin{equation}
		\label{eq:queue-pfp}
		\FP_{\mathcal Q, m}^w = 1 - \left(1 - \FP_{\mathcal F,c}\right)^{L-1} \left(1 - \FP_{\mathcal F, m \bmod c }\right)
	$$ %\end{equation}
	where $\bmod$ is the modulo operator and 
	 $\FP_{\mathcal F,m}$ is the probability of false positive of a subfilter $\mathcal F$ after $m$ insertions.
\end{restatable}

%\begin{proof}
%	Let $E = (e_1, \dotsc, e_m, \dotsc)$ be a stream and $e^\star \notwdup E$. 
%	
%	Therefore, $e^\star$ is a false positive if and only if at least one subquery $\mathcal F_i.\mathsf{Lookup(e^\star)}$ returns \DUPLICATE. Conversely, $e^\star$ is \emph{not} a false positive when all subqueries $\mathcal F_i.\mathsf{Lookup(e^\star)}$ return \UNSEEN, i.e., when $e^\star$ is not a false positive for each subfilter.
%	
%	Each subfilter has undergone $c$ insertions, safe for the first subfilter which has only undergone $m \bmod c$, we immediately get Eq.~(\ref{eq:queue-pfp}).
%\end{proof}

\paragraph{Remark.} In the case $w < cL$, as mentioned previously, there is a non-zero probability that $e^\star$ is in the last subfilter's memory, despite not belonging to the sliding window. 
	
Assuming a uniformly random input stream, and writing $\delta = cL - w$, the probability that $e^\star$ has occurred in $\{e_{m-cL}, \dotsc e_{m-w+1}\}$ is $1-\left(1-\frac1{|\Gamma|}\right)^\delta$. For large $|\Gamma|$ (as is expected to be the case in most applications), this probability is about $\frac{\delta}{|\Gamma|} \ll 1$. Hence, we can neglect the probability that $e^\star$ is present in the filter, and we consider the result of Theorem~\ref{thm:queue-pfp} to be a very good approximation even when $w < cL$.

\subsubsection{Probability of False Negative}\label{sub:fnr}
\begin{restatable}{thm}{pfnQueue}
	\label{thm:queue-pfn}
	Let $\FN_{\mathcal Q, m}^w$ be the probability of false negative of $\mathcal Q$ after $m > w$ insertions on a sliding window of size $w = cL$, we have
	$$%\begin{equation}
	\FN_{\mathcal Q, m}^w 
	=  u_c^{L-1} u_{m \bmod c}
	$$%\end{equation}
	where $\bmod$ is the modulo operator, and we have introduced the short-hand notation $u_\eta= p_{\eta}\FN_{\mathcal F,\eta} + \left(1-p_{\eta}\right)\left(1-\FP_{\mathcal F,\eta}\right)$ where 
	$\FN_{\mathcal F,\eta}$ (resp. $\FP_{\mathcal F, \eta}$) is the probability of false negative (resp. false positive) of the subfilter $\mathcal F$ after $\eta$ insertions, and
	$%\begin{equation*}
		p_\eta = \frac{1-\left(1-\frac 1{|\Gamma|}\right)^\eta}{1-\left(1-\frac 1{|\Gamma|}\right)^w} \approx \frac{\eta}{w}.
	$%\end{equation*}
\end{restatable}

%\begin{proof}
%	Let $E = \{e_1, \dotsc, e_m, \dotsc\}$ be a stream, let $w$ be a sliding window and let $e^\star \wdup E$. 
%	
%	Then $e^\star$ is a false negative if and only if all subfilters $\mathcal F_i$ answer $\mathcal F_i.\mathsf{Detect}(e^\star) = \UNSEEN$. There can be two cases:
%	\begin{itemize}
%		\item $e^\star$ is present in $\mathcal F_i$'s sub-sliding window;
%		\item $e^\star$ is not present in $\mathcal F_i$'s sub-sliding window.
%	\end{itemize}
%In the first case, $\mathcal F_i.\mathsf{Detect}(e^\star)$ returns \UNSEEN if and only if $e^\star$ is a false negative for $\mathcal F_i$. This happens with probability $\FN_{\mathcal F,c}$ by definition, except for $\mathcal F_0$, for which the probability is $\FN_{\mathcal F, m\bmod c}$.
%
%In the second case, $\mathcal F_i.\mathsf{Detect}(e^\star)$ returns \UNSEEN if and only if $e^\star$ is not a false positive for $\mathcal F_i$, which happens with probability $1-\FP_{\mathcal F,c}$, execpt for $\mathcal F_0$, for which the probability is $1-\FP_{\mathcal F, m\bmod c}$.
%
%Finally, each event is weighted by the probability $p_c$ that $e^\star$ is in $\mathcal F_i$'s sub-sliding window:
%\begin{align*}
%	p_c 
%	& = \Pr[\text{$e^\star$ is in }\mathcal F_i \text{ sub-sliding window | } e^\star \wdup E]\\
%	& = \frac{\Pr[\text{$e^\star$ is in }\mathcal F_i \text{ sub-sliding window } \cap e^\star \wdup E]}{\Pr[e^\star \wdup E]}\\
%	& = \frac{\Pr[\text{$e^\star$ is in }\mathcal F_i \text{ sub-sliding window}]}{\Pr[e^\star \wdup E]}\\
%%		\end{align*}\begin{align*}p_c
%	& = \frac{1 - \Pr[\text{$e^\star$ is not in }\mathcal F_i \text{ sub-sliding window }]}{1 - \Pr[e^\star \notwdup E]}\\
%	& = \frac{1 - \left(1 - \frac 1{|\Gamma|}\right)^c}{1 - \left(1-\frac{1}{|\Gamma|}\right)^w}
%\end{align*}
%This concludes the proof.
%\end{proof}
\paragraph{Remark.} As previously, the effect of $w < cL$ is negligible for all practical purposes and Theorem~\ref{thm:queue-pfn} is considered a good approximation in that regime. 

\subsection{FNR and FPR}
From the above expressions we can derive relatively compact explicit formulas for the queuing filter's FPR and FNR when $m = cn$ for $n$ a positive integer.
\begin{restatable}{thm}{fprQueue}
	Let $\FPR_{\mathcal Q, m}^w$ be the false positive rate of $\mathcal Q$ after $m=cn > w$ insertions on a sliding window of size $w = cL$, we have
	\begin{equation*}
	\FPR_{\mathcal Q, cn}^w = 1 -  \frac{(1 - \FP_{\mathcal F, c})^{L-1}}{c}\sum_{\ell = 0}^{c-1} (1-\FP_{\mathcal F, \ell}).
	\end{equation*}
\end{restatable}
%\begin{proof}
%
%\begin{align*}
%	\FPR_{\mathcal Q, cn}^w
%	& = \frac{1}{cn} \sum_{k=1}^{cn} \FP_{\mathcal Q, k}^w 
%	 = \frac{1}{cn} \sum_{k=1}^{n} \sum_{\ell = 0}^{c-1} \FP_{\mathcal Q, k + \ell}^w 
%	 = \frac1c \sum_{\ell = 0}^{c-1} \FP_{\mathcal Q, \ell}^w \\
%	& = \frac1c \sum_{\ell = 0}^{c-1}  1 - (1 - \FP_{\mathcal F, c})^{L-1}(1 - \FP_{\mathcal F}, \ell) \\
%	& = 1 - \frac1c(1 - \FP_{\mathcal F, c})^{L-1}\sum_{\ell = 0}^{c-1} (1-\FP_{\mathcal F, \ell})
%\end{align*}
%\end{proof}

\begin{restatable}{thm}{fnrQueue}
	Let $\FNR_{\mathcal Q, m}^w$ be the false negative rate of $\mathcal Q$ after $m=cn > w$ insertions on a sliding window of size $w = cL$, we have
	\begin{equation*}
	\FNR_{\mathcal Q, cn}^w = \frac{u_c^{L-1}}{c}  \sum_{\ell = 0}^{c-1}u_\ell.
	\end{equation*}
\end{restatable}

%\begin{proof}
%\begin{align*}
%	\FNR_{\mathcal Q, cn}^w 
%	& = \frac{1}{cn} \sum_{k=1}^{cn} \FN_{\mathcal Q, k}^w 
%	 = \frac{1}{cn} \sum_{k=1}^{n} \sum_{\ell = 0}^{c-1} \FN_{\mathcal Q, k + \ell}^w 
%	= \frac1c \sum_{\ell = 0}^{c-1} \FN_{\mathcal Q, \ell}^w \\
%	& = \frac1c \sum_{\ell = 0}^{c-1} u_c^{L-1} u_{\ell} 
%	 = \frac{u_c^{L-1}}{c}  \sum_{\ell = 0}^{c-1}u_\ell
%\end{align*}
%\end{proof}
As for the probabilities, the expressions derived above for the FNR and FNR are valid to first order in $(w - cL)/|\Gamma|$, i.e. they are good approximations even when $w \approx cL$. 

\subsection{Optimising queuing filters}
\label{sec:optimising}
Let us relax, temporarily, the a priori constraint that $w = cL$. The parameter $L$ determines how many subfilters appear in the queuing construction. Summing up the false positive and false negative rates, we have a total error rate 
$	\operatorname{ER}_{\mathcal Q, cn}^{w}
	= 1 - \alpha \beta^{L-1} + \alpha' \beta'^{L-1}
$,
where $\beta = 1 - \FP_{\mathcal F, c}$, $\beta' = u_c$, 
$\alpha = \frac1c \sum_{\ell = 0}^{c-1}1 - \FP_{\mathcal F, \ell}$ and $\alpha' = \frac1c \sum_{\ell = 0}^{c-1} u_\ell$ depend on $w$, $c$ and the choice of subfilter type $\mathcal F$.

Because $u_\eta = p_{\eta}\FN_{\mathcal F,\eta} + \left(1-p_{\eta}\right)\left(1-\FP_{\mathcal F,\eta}\right)$, differentiating with respect to $L$, knowing that $w = Lc$, and equating the derivative to $0$, one can find the optimal value for $L$ by solving for $x$, which has been obtained via Mathematica:
%\begin{equation*}

\begin{align*}
&-\alpha \beta^{-1+x} \log(\beta) 
+ \left(\beta + \FN_{\mathcal F, c} (-1+x)\right)^{-2+x} x^{-x} \left[\vphantom{\left(\FN^{\FN}\right)}\right. 
	-\alpha \beta + \FN_{\mathcal F, c} \left(-\beta (-2+x)+ \FN_{\mathcal F, c} (-1+x)\right) \\
&\quad	+ \left(\alpha+ \FN_{\mathcal F, c} (-1+x)\right)
\times\left(\beta + \FN_{\mathcal F, c} (-1+x)\right) \left(\log\left(\beta + \FN_{\mathcal F, c} (-1+x)\right)-\log(x)\right)
\left.\vphantom{\left(\FN^{\FN}\right)}\right] = 0
\end{align*}

If numerically solving the equation for individual cases is feasible, it seems unlikely that a closed-form formula exists.
%\end{equation*}

%Differentiating with respect to $L$ and equating to zero gives an optimum
%\begin{equation*}
%	L_{c, \mathcal F} = -\log((\alpha \beta' \log(\beta))/(\alpha'\beta \log(\beta')))/\log(\beta/\beta').
%\end{equation*} 
%Thus for a given $c$ and choice of subfilter $\mathcal F$ we get an optimal value of $L = L_{c, \mathcal F}$,
%and we set, \emph{a posteriori}, $w \gets cL$.
%
%If instead $w$ is a given value, solving $w = c L_{c, \mathcal F}$ analytically may be challenging.
%A practical solution is to interpolate between explicitly computed values $cL_{c, \mathcal F}$ to find the 
%closest match for the target $w$.
%
%\todo[À vérifier la suite je suis pas convaincu encore]
%\begin{theorem}
%	In first approximation, a low value of $L$ for constant value $cL$ leads to a higher false negative probability for the queuing filter.
%\end{theorem}
%
%\begin{proof}
%As a matter of fact, on average the queuing filter consists of $L-1$ subfilters with $c$ insertions, and one subfilter with $\frac c2$ insertions. As such, we can say that the queuing filter is a representation of the last $(c-\frac12) L$ last elements. 
%
%Let us consider $e^\star \wdup E$. If, in the sliding window, all duplicates only happened at least $(c-\frac12)L$ epochs ago (which happens with probability $\left(1-\frac1{|\Gamma|}\right)^{(c-\frac12)L}$), we know that no duplicates happened in any subfilter's sliding window. By using the same reasoning as Theorem~\ref{thm:queue-pfn}'s proof, we get a probability of false negative of around $\left[0 + \left(1-\frac1{|\Gamma|}\right)^{(c-\frac12)L}(1-\FP_{\mathcal F,c})\right]^L$. Given that $cL \approx w$ which is a constant, we get that the probability of false positive mostly varies, in this case, with $\left(1-\FP_{\mathcal F,c}\right)^L$. 
%
%Hence, the lower $L$ is, the higher the probability of false negative will be.
%
%The approximations made in this proof are backed by experimental results.
%\end{proof}


\subsection{Queuing filters from existing DDFs}
\label{sec:subfilter-select}
\label{sec:optimal_ddf}
Our queuing construction relies on a choice of subfilters. A first observation 
is that we may assume that all subfilters can be instances of a single 
DDF design (rather than a combination of different designs).

Indeed, a simple symmetry argument shows that a heterogenous selection of subfilters is always worse than a homogeneous one: the crux is that all subfilters play the same role in turn. Therefore we lose nothing by replacing atomically one subfilter by a more efficient one. Applying this to each subfilter we end up with a homogenous selection.

It remains to decide which subfilter construction to choose. The results of an experimental comparison of different DDFs (details about the benchmark are given in Section~\ref{sub:saturation}) are summarized in Figure~\ref{fig:graph_n}. It appears that the most efficient filter (in terms of saturation rate) is the QHT, from \cite{10.1145/3297280.3297335}.

%As noted in Section~\ref{sub:related}, A2 filters are a primitive form of a queuing filter (with $L=2$ and Bloom filters as subfilters) and thus were not included in this benchmark, as we focused on adapting filters designed for DDP to the sliding window DDP. This raises the question of whether recursive constructions (queuing filters of queuing filters) have particularly interesting properties, which we leave open for future work. 

%In the experiments we use QHT for the benchmarking of the queuing construction.

\section{Experiments and Benchmarks}\label{sec:bench}

This section provides details and additional information on the benchmarking experiments run to validate the above analysis. All code is accessible online and will be disclosed after peer review.

\paragraph{Benchmarking queuing filters.}
Applying the queuing construction to DDFs from the literature, we get new filters which are compared in the wDDP setting.

In Section~\ref{sec:optimal_ddf} we suggested the heuristic that the DDFs with the least saturation rate in the DDP would yield the best (error-wise) queuing filter for the wDDP. This heuristic is supported by results, summarized in Figure~\ref{tab:comparing_queuing}. For this benchmark we used the following parameters: uniform stream from an alphabet of size $|\Gamma| = 2^{18}$, memory size $M=100,000$ bits, sliding window of size $w = 10,000$, and we measure the error rate (sum of $\FNR^w$ and $\FPR^w$).

A very interesting observation is that $Lw$ approaches the size of the stream, there is a drop in the error. This is an artifact due to the finite size of our simulations; the stream should be considered infinite, and this drop disappears as the simulation is run for longer (see Appendix~\ref{app:finite}). This effect also alters the error rates for smaller window sizes, albeit much less, and we expect that filter designers care primarily about the small window regime. Nevertheless a complete understanding of this effect would be of theoretical interest, and we leave the study of this phenomenon for future work.


\begin{figure}[t]
\makebox[\textwidth]{\makebox[1.15\textwidth]{%
	\begin{minipage}{.5\textwidth}
		\centering
		\begin{filecontents*}[overwrite]{qht.dat}
w	 FPR 	 FNR 	 Error 	 1-E 	 Phi 	 percentDup
1000 	3.97 	5.87 	9.84 	90.16 	27.42 	0.38
2000 	7.63 	6.91 	14.54 	85.46 	26.98 	0.76
3000 	11.07 	7.51 	18.57 	81.43 	26.6 	1.14
5000 	17.26 	9 	26.25 	73.75 	25.86 	1.9
7000 	22.68 	9.76 	32.44 	67.56 	25.22 	2.65
10000 	29.65 	11.44 	41.09 	58.91 	23.98 	3.74
15000 	38.85 	12.91 	51.76 	48.24 	22.39 	5.54
20000 	45.73 	13.77 	59.5 	40.5 	21.05 	7.28
30000 	55.12 	14.91 	70.03 	29.97 	18.74 	10.64
50000 	64.51 	15.82 	80.33 	19.67 	15.8 	16.96
70000 	68.47 	16.52 	84.99 	15.01 	13.97 	22.62
80000 	69.45 	16.81 	86.26 	13.74 	13.44 	25.29
100000 	70.39 	17.49 	87.89 	12.11 	12.69 	30.18
125000 	70.35 	18.22 	88.57 	11.43 	12.55 	35.72
200000 	67.66 	20.23 	87.89 	12.11 	13.73 	48.72
300000 	61.89 	22.42 	84.31 	15.69 	17 	59.89
400000 	55.45 	24.47 	79.92 	20.08 	20.46 	66.49
500000 	49.39 	26.28 	75.66 	24.34 	23.56 	70.29
\end{filecontents*}


\begin{filecontents*}[overwrite]{cuckoo.dat}
w	 FPR 	 FNR 	 Error 	 Phi 	 percentDup
1000 	84.26 	0.73 	84.99 	15.01 	2.55 	0.38
2000 	97.47 	0.1 	97.57 	2.43 	1.35 	0.76
3000 	99.55 	0.01 	99.56 	0.44 	0.71 	1.14
5000 	99.94 	0 	99.94 	0.06 	0.35 	1.9
7000 	99.95 	0 	99.95 	0.05 	0.37 	2.65
10000 	99.95 	0 	99.95 	0.05 	0.45 	3.74
15000 	99.95 	0 	99.95 	0.05 	0.55 	5.54
20000 	99.94 	0 	99.94 	0.06 	0.63 	7.28
30000 	99.94 	0 	99.94 	0.06 	0.78 	10.64
50000 	99.94 	0 	99.94 	0.06 	1.02 	16.96
70000 	99.93 	0 	99.93 	0.07 	1.22 	22.62
80000 	99.93 	0 	99.93 	0.07 	1.32 	25.29
100000 	99.93 	0 	99.93 	0.07 	1.49 	30.18
125000 	99.92 	0 	99.92 	0.08 	1.69 	35.72
200000 	99.9 	0 	99.9 	0.1 	2.21 	48.72
300000 	99.87 	0 	99.87 	0.13 	2.77 	59.89
400000 	99.85 	0 	99.85 	0.15 	3.19 	66.49
500000 	99.83 	0 	99.83 	0.17 	3.48 	70.29
\end{filecontents*}

\begin{filecontents*}[overwrite]{sqf.dat}
w	 FPR 	 FNR 	 Error 	 1-E 	 Phi 	 percentDup
1000 	10.72 	6.01 	16.73 	83.27 	16.41 	0.38
2000 	19.97 	6.7 	26.66 	73.34 	15.8 	0.76
3000 	27.88 	7.22 	35.1 	64.9 	15.24 	1.14
5000 	40.71 	8.05 	48.76 	51.24 	14.2 	1.9
7000 	50.52 	7.9 	58.42 	41.58 	13.35 	2.65
10000 	61.18 	8.07 	69.25 	30.75 	12.04 	3.74
15000 	72.32 	7.74 	80.06 	19.94 	10.32 	5.54
20000 	78.82 	7.14 	85.96 	14.04 	9.09 	7.28
30000 	85.32 	6.37 	91.69 	8.31 	7.43 	10.64
50000 	89.56 	5.62 	95.18 	4.82 	6.14 	16.96
70000 	90.34 	5.54 	95.88 	4.12 	6.1 	22.62
80000 	90.28 	5.63 	95.92 	4.08 	6.3 	25.29
100000 	90.07 	5.7 	95.77 	4.23 	6.91 	30.18
125000 	89.19 	6.06 	95.26 	4.74 	7.9 	35.72
200000 	85.84 	6.9 	92.75 	7.25 	11.77 	48.72
300000 	80.2 	8.09 	88.29 	11.71 	17.19 	59.89
400000 	73.85 	9.29 	83.14 	16.86 	22.32 	66.49
500000 	67.61 	10.94 	78.55 	21.45 	25.91 	70.29
\end{filecontents*}

\begin{filecontents*}[overwrite]{sbf.dat}
	w	 FPR 	 FNR 	 Error 	 1-E 	 Phi 	 percentDup
1000 	52.75 	2.49 	55.24 	44.76 	5.51 	0.38
2000 	75.9 	1.46 	77.35 	22.65 	4.61 	0.76
3000 	85.81 	1.23 	87.03 	12.97 	3.95 	1.13
5000 	92.44 	1.6 	94.04 	5.96 	3.09 	1.89
7000 	94.14 	2.05 	96.18 	3.82 	2.62 	2.63
10000 	94.79 	2.7 	97.49 	2.51 	2.16 	3.74
15000 	94.98 	3.34 	98.32 	1.68 	1.78 	5.55
20000 	95.04 	3.68 	98.73 	1.27 	1.54 	7.33
30000 	95.08 	4.01 	99.08 	0.92 	1.32 	10.8
50000 	95.08 	4.32 	99.4 	0.6 	1.06 	17.33
70000 	95.06 	4.43 	99.48 	0.52 	1.02 	23.36
80000 	95.02 	4.47 	99.49 	0.51 	1.04 	26.2
100000 	94.96 	4.51 	99.47 	0.53 	1.14 	31.56
125000 	94.87 	4.58 	99.46 	0.54 	1.22 	37.7
200000 	94.49 	4.68 	99.17 	0.83 	1.9 	52.89
300000 	93.69 	4.81 	98.5 	1.5 	3.14 	67.33
400000 	92.52 	4.91 	97.42 	2.58 	4.75 	77.08
500000 	90.79 	5.03 	95.82 	4.18 	6.65 	83.67
\end{filecontents*}

\begin{tikzpicture}
\begin{semilogxaxis}[legend cell align = left, width=0.4\textwidth, legend pos = outer north east, ylabel=$\FPR^w + \FNR^w (\times 100)$, xlabel=$w$]
  	\addplot[color=red, mark=+] table[x=w,y=Error] {qht.dat};
\addlegendentry{Q\_QHT}
  	\addplot[color=blue, mark=x] table[x=w,y=Error] {sqf.dat};
\addlegendentry{Q\_SQF}

\addplot[color=olive, mark=square, mark options = {scale=0.7}] table[x=w,y=Error] {sbf.dat};
\addlegendentry{Q\_SBF}

  	\addplot[color=teal, mark=triangle] table[x=w,y=Error] {cuckoo.dat};
\addlegendentry{Q\_Cuckoo}


\end{semilogxaxis}

\end{tikzpicture}
		\caption{Error rate (times 100) of queuing filters as a function of window size, $M=10^5$, $L=10$, $|\Gamma| = 2^{18}$, on a stream of size $10^7$.}\label{tab:comparing_queuing}
	\end{minipage}
	\hfill
	\begin{minipage}{.6\textwidth}
		\centering
			
	% Stream length = 100000
	\begin{filecontents*}[overwrite]{100.dat}
L	 FPR 	 FNR 	 Error 	 1-E 	 Phi 	 percentDup
1 	0.02 	51.95 	51.97 	48.03 	47.97 	0.04
2 	0.06 	26.59 	26.65 	73.35 	48.77 	0.04
3 	0.1 	14.63 	14.74 	85.26 	46.33 	0.04
5 	0.2 	12.68 	12.88 	87.12 	36.49 	0.04
10 	0.41 	7.07 	7.48 	92.52 	28.03 	0.04
21 	0.91 	0 	0.91 	99.09 	20.71 	0.04
31 	1.6 	0.24 	1.85 	98.15 	15.63 	0.04
41 	2.12 	0.24 	2.37 	97.63 	13.59 	0.04
50 	2.1 	0.73 	2.83 	97.17 	13.58 	0.04
100 	4.19 	0.49 	4.68 	95.32 	9.59 	0.04
\end{filecontents*}
	\begin{filecontents*}[overwrite]{500.dat}
L	 FPR 	 FNR 	 Error 	 1-E 	 Phi 	 percentDup
1 	0.11 	50.02 	50.13 	49.87 	49.2 	0.2
2 	0.31 	25.99 	26.3 	73.7 	48.61 	0.2
3 	0.52 	17.19 	17.71 	82.29 	44.53 	0.2
5 	0.97 	9.85 	10.81 	89.19 	37.47 	0.2
10 	2.02 	5.6 	7.62 	92.38 	28.13 	0.2
21 	4.27 	2.5 	6.77 	93.23 	20.2 	0.2
31 	6.61 	0.45 	7.06 	92.94 	16.5 	0.2
41 	8.78 	0.65 	9.43 	90.57 	14.17 	0.2
50 	9.96 	2.15 	12.11 	87.89 	13.02 	0.2
100 	19.1 	0.85 	19.95 	80.05 	9.07 	0.2
\end{filecontents*}

	\begin{filecontents*}[overwrite]{1000.dat}
		L	 FPR 	 FNR 	 Error 	 1-E 	 Phi 	 percentDup
1 	0.22 	48.64 	48.86 	51.14 	49.32 	0.38
2 	0.62 	24.36 	24.98 	75.02 	48.79 	0.38
3 	1.05 	16.76 	17.82 	82.18 	43.68 	0.38
5 	1.91 	10.23 	12.14 	87.86 	36.64 	0.38
10 	3.97 	5.87 	9.84 	90.16 	27.42 	0.38
21 	8.33 	2.3 	10.63 	89.37 	19.61 	0.38
31 	12.36 	1.17 	13.53 	86.47 	16.05 	0.38
41 	16.12 	0.97 	17.08 	82.92 	13.82 	0.38
50 	18.84 	1.54 	20.38 	79.62 	12.5 	0.38
100 	34.31 	1.07 	35.38 	64.62 	8.39 	0.38
\end{filecontents*}

	\begin{filecontents*}[overwrite]{5000.dat}
L	 FPR 	 FNR 	 Error 	 1-E 	 Phi 	 percentDup
1 	1.03 	51.95 	52.98 	47.02 	46.8 	1.9
2 	2.96 	28.53 	31.49 	68.51 	46.36 	1.9
3 	4.89 	21.11 	26 	74 	41.6 	1.9
5 	8.61 	14.27 	22.88 	77.12 	35 	1.9
10 	17.26 	9 	26.25 	73.75 	25.86 	1.9
21 	33.5 	5.19 	38.69 	61.31 	17.6 	1.9
31 	45.58 	4.23 	49.81 	50.19 	13.75 	1.9
41 	55.28 	3.29 	58.58 	41.42 	11.4 	1.9
50 	62.65 	2.76 	65.41 	34.59 	9.81 	1.9
100 	86.13 	0.84 	86.97 	13.03 	5.19 	1.9
\end{filecontents*}
	\begin{filecontents*}[overwrite]{10000.dat}
		L	 FPR 	 FNR 	 Error 	 1-E 	 Phi 	 percentDup
1 	1.94 	52.55 	54.48 	45.52 	46.13 	3.74
2 	5.44 	30.63 	36.08 	63.92 	45.13 	3.74
3 	8.96 	23.3 	32.26 	67.74 	40.29 	3.74
5 	15.41 	16.95 	32.36 	67.64 	33.44 	3.74
10 	29.65 	11.44 	41.09 	58.91 	23.98 	3.74
21 	53.13 	6.43 	59.56 	40.44 	15.41 	3.74
31 	67.63 	4.21 	71.84 	28.16 	11.52 	3.74
41 	77.48 	2.98 	80.46 	19.54 	8.98 	3.74
50 	83.75 	1.85 	85.6 	14.4 	7.51 	3.74
100 	97.28 	0.28 	97.56 	2.44 	2.89 	3.74
\end{filecontents*}


\begin{filecontents*}[overwrite]{30000.dat}
L	 FPR 	 FNR 	 Error 	 1-E 	 Phi 	 percentDup
1 	4.72 	56.97 	61.69 	38.31 	41.7 	10.64
2 	12.38 	39.5 	51.88 	48.12 	39.05 	10.64
3 	19.48 	32.5 	51.98 	48.02 	34.38 	10.64
5 	31.88 	24.6 	56.48 	43.52 	27.87 	10.64
10 	55.12 	14.91 	70.03 	29.97 	18.74 	10.64
21 	81.79 	5.66 	87.45 	12.55 	10.33 	10.64
31 	91.72 	2.44 	94.16 	5.84 	6.77 	10.64
41 	96.03 	1.03 	97.06 	2.94 	4.82 	10.64
50 	97.86 	0.5 	98.36 	1.64 	3.63 	10.64
100 	99.61 	0.02 	99.63 	0.37 	1.91 	10.64
\end{filecontents*}

	\begin{filecontents*}[overwrite]{50000.dat}
L	 FPR 	 FNR 	 Error 	 1-E 	 Phi 	 percentDup
1 	6.69 	60.24 	66.93 	33.07 	37.79 	16.96
2 	16.31 	45.21 	61.52 	38.48 	34.4 	16.96
3 	24.88 	38 	62.88 	37.12 	30.08 	16.96
5 	39.48 	28.64 	68.13 	31.87 	24.05 	16.96
10 	64.51 	15.82 	80.33 	19.67 	15.8 	16.96
21 	88.45 	4.63 	93.08 	6.92 	8.52 	16.96
31 	95.38 	1.59 	96.97 	3.03 	5.73 	16.96
41 	97.77 	0.58 	98.35 	1.65 	4.47 	16.96
50 	98.64 	0.24 	98.87 	1.13 	3.93 	16.96
100 	99.5 	0.01 	99.51 	0.49 	2.83 	16.96
\end{filecontents*}
	
	\begin{filecontents*}[overwrite]{100000.dat}
		L	 FPR 	 FNR 	 Error 	 1-E 	 Phi 	 percentDup
1 	9.39 	65.08 	74.47 	25.53 	31.13 	30.18
2 	20.34 	53.1 	73.44 	26.56 	27.05 	30.18
3 	29.86 	45.5 	75.36 	24.64 	23.39 	30.18
5 	45.55 	34.31 	79.87 	20.13 	18.49 	30.18
10 	70.39 	17.49 	87.89 	12.11 	12.69 	30.18
21 	90.81 	4.23 	95.04 	4.96 	8.54 	30.18
31 	95.87 	1.23 	97.1 	2.9 	7.51 	30.18
41 	97.46 	0.45 	97.9 	2.1 	7.03 	30.18
50 	98.08 	0.21 	98.29 	1.71 	6.66 	30.18
100 	99.08 	0.04 	99.11 	0.89 	5.04 	30.18
\end{filecontents*}
	
	
	\begin{tikzpicture}
	\begin{semilogxaxis}[legend cell align = right, width=0.35\textwidth, legend pos = outer north east, ylabel= $\FPR^w + \FNR^w (\times 100)$, xlabel = L]
%	\addplot table[x=L,y=1-E] {100.dat};
%	\addlegendentry{w = 100}
%	\addplot table[x=L,y=1-E] {500.dat};
%	\addlegendentry{w = 500}
   \addlegendimage{empty legend}
   \addlegendentry{$w$ value\hspace{.6cm}}
   
	\addplot table[x=L,y=Error] {1000.dat};
	\addlegendentry{$10^3$}	
	\addplot table[x=L,y=Error] {5000.dat};
	\addlegendentry{$5\cdot 10^3$}
	\addplot[mark=triangle] table[x=L,y=Error] {10000.dat};
	\addlegendentry{$10^4$}
	\addplot table[x=L,y=Error] {30000.dat};
	\addlegendentry{$3\cdot 10^4$}
	\addplot table[x=L,y=Error] {50000.dat};
	\addlegendentry{$5\cdot 10^4$}
	\addplot table[x=L,y=Error] {100000.dat};
	\addlegendentry{$10^5$}
	\end{semilogxaxis}
	\end{tikzpicture}

		\caption{Evolution of the error rate of a queueing QHT as a function of $L$, for several window sizes, with $M=10^5$, $|\Gamma| = 2^{18}$, on a stream of size $10^6$.}\label{fig:bench_l}
	\end{minipage}
}}
\end{figure}

%\begin{table}[]
%	\begin{tabular}{rccccc}
%			   & $w=1000$ & $w=5000$ & $w=10^4$ & $w=10^5$ & $w=10^6$ \\ \hline
%		QHT    & \textbf{9.84} 	  & \textbf{26.25}	 & \textbf{41.09}	 & \textbf{87.89}	  & 70.76      \\
%		Cuckoo & 84.99	  & 99.94	 & 99.95	 & 99.93	  & 99.80      \\
%		SQF    & 16.73 	  & 48.76	 & 69.25	 & 95.77 	  & \textbf{70.14}     \\
%		SBF    & 54.88 	  & 93.93	 & 97.33	 & 97.41	  & 71.17      \\
%	\end{tabular}
%	\caption{Error rate (times 100) of various queuing filters on various sliding window sizes, $M=100000$, $L=10$, $|\Gamma| = 2^{16}$. For each sliding window, the best performance is written in bold.}\label{tab:comparing_queuing}
%\end{table}

\paragraph{The number of subfilters}\label{sub:L_bench}
The number of subfilters $L$ is an important parameter in the queuing construction, as it affects the filter's error rate in a nontrivial way. An illustration of this dependence is shown in Figure~\ref{fig:bench_l} which plots the error rate of a queueing QQHTD on an uniform stream of alphabet size $\Gamma = 2^{16}$, with $10^5$ elements in the stream, on various sliding window sizes. 

We observe that the optimal value for $L$ does indeed depend on the desired sliding window. However, other experiments on alphabets of other sizes yield very similar results, hence validating the observation made in Section~\ref{sec:optimising} that the optimal number of subfilters does not depends on the alphabet, at least in first approximation.

\paragraph{Filters vs queued filters.}
Using the same stream as previously, we can build queued filters (with an optimal value $L$ for each considered sliding window) and compare their performances to that of non-modified filters. Results on the QHT and SQF are shown in Figure~\ref{fig:improvement}, results for the Cuckoo and SBF are shown in Appendix~\ref{app:improvement}.

\begin{figure}[t]
	\centering
	\begin{filecontents*}[overwrite]{Queueing_QHT.dat}
	w	 FPR 	 FNR 	 Error 	 1-E 	 Phi 	 percentDup
	100 	0.11 	15.57 	15.68 	84.32 	43.88 	0.04
	500 	0.53 	17.47 	18. 	82. 	43.34 	0.19
	1000 	1.05 	17.19 	18.24 	81.76 	43.4 	0.38
	5000 	4.89 	20.28 	25.18 	74.82 	41.87 	1.89
	10000 	8.95 	23.49 	32.44 	67.56 	40.2 	3.74%
%	15000 	12.36 	26.22 	38.58 	61.42 	38.6 	5.55
%	20000 	15.21 	28.52 	43.74 	56.26 	37.13 	7.33
%	30000 	19.71 	32.52 	52.22 	47.78 	34.3 	10.8
%	50000 	25.42 	37.95 	63.37 	36.63 	29.77 	17.33
%	70000 	28.62 	41.77 	70.38 	29.62 	26.18 	23.36
%	80000 	29.69 	43.24 	72.93 	27.07 	24.69 	26.2
%	100000 	31.15 	45.61 	76.76 	23.24 	22.2 	31.56
%	125000 	32.25 	47.91 	80.17 	19.83 	19.64 	37.7
%	200000 	33.51 	51.82 	85.33 	14.67 	14.88 	52.89
%	300000 	33.75 	54.21 	87.96 	12.04 	11.45 	67.33
%	400000 	33.55 	55.44 	88.99 	11.01 	9.38 	77.08
%	500000 	32.93 	56.11 	89.04 	10.96 	8.21 	83.67
%	
%	% L = 2
	15000 	7.67 	33.54 	41.21 	58.79 	43.14 	5.55
	20000 	9.54 	35.76 	45.3 	54.7 	41.66 	7.33
	30000 	12.54 	39.68 	52.22 	47.78 	38.85 	10.8
	50000 	16.57 	45.38 	61.95 	38.05 	34.13 	17.33
	70000 	18.99 	49.46 	68.45 	31.55 	30.3 	23.36
	80000 	19.84 	51.05 	70.88 	29.12 	28.69 	26.2
	100000 	21.07 	53.66 	74.73 	25.27 	25.87 	31.56
	125000 	22.04 	56.02 	78.06 	21.94 	23.14 	37.7
%%	200000 	23.38 	60.22 	83.6 	16.4 	17.54 	52.89
%%	300000 	23.92 	62.78 	86.7 	13.3 	13.27 	67.33
%%	400000 	23.94 	64.01 	87.95 	12.05 	10.75 	77.08
%%	500000 	23.66 	64.68 	88.34 	11.66 	9.14 	83.67
%%	L=1
	200000 	11.6 	70.95 	82.55 	17.45 	21.45 	52.89
	300000 	12.42 	73.04 	85.46 	14.54 	16.41 	67.33
	400000 	12.87 	73.99 	86.86 	13.14 	13.12 	77.08
	500000 	13.07 	74.53 	87.6 	12.4 	10.82 	83.67
\end{filecontents*}

\begin{filecontents*}[overwrite]{QHT.dat}
	w	 FPR 	 FNR 	 Error 	 1-E 	 Phi 	 percentDup
	100 	25.14 	0.05 	25.19 	74.81 	3.36 	0.04
	500 	25.03 	0.65 	25.67 	74.33 	7.47 	0.19
	1000 	24.89 	1.16 	26.05 	73.95 	10.47 	0.38
	5000 	23.85 	6.05 	29.9 	70.1 	21.97 	1.89
	10000 	22.71 	11.4 	34.11 	65.89 	28.79 	3.74
	15000 	21.73 	16.23 	37.96 	62.04 	32.74 	5.55
	20000 	20.88 	20.56 	41.44 	58.56 	35.17 	7.33
	30000 	19.49 	27.93 	47.42 	52.58 	37.6 	10.8
	50000 	17.64 	38.88 	56.52 	43.48 	37.92 	17.33
	70000 	16.53 	46.48 	63.01 	36.99 	36.06 	23.36
	80000 	16.15 	49.42 	65.57 	34.43 	34.88 	26.2
	100000 	15.59 	54.06 	69.65 	30.35 	32.5 	31.56
	125000 	15.16 	58.29 	73.45 	26.55 	29.65 	37.7
	200000 	14.6 	65.41 	80.01 	19.99 	22.99 	52.89
	300000 	14.36 	69.58 	83.95 	16.05 	17.35 	67.33
	400000 	14.24 	71.58 	85.82 	14.18 	13.73 	77.08
	500000 	14.11 	72.67 	86.78 	13.22 	11.26 	83.67
\end{filecontents*}

\begin{filecontents*}[overwrite]{Queueing_Cuckoo.dat}
	w	 FPR 	 FNR 	 Error 	 1-E 	 Phi 	 percentDup
	100 	4.04 	15.05 	19.09 	80.91 	7.97 	0.04
	500 	18.4 	14.19 	32.59 	67.41 	7.57 	0.19
%	1000 	33.37 	11.31 	44.68 	55.32 	7.2 	0.38
%	5000 	86.35 	2.56 	88.91 	11.09 	4.42 	1.89
%	10000 	97.96 	0.41 	98.37 	1.63 	2.22 	3.74
%	15000 	99.65 	0.07 	99.73 	0.27 	1.09 	5.55
%	20000 	99.92 	0.02 	99.94 	0.06 	0.6 	7.33
%	30000 	99.98 	0 	99.98 	0.02 	0.51 	10.8
%	50000 	99.98 	0 	99.98 	0.02 	0.66 	17.33
%	70000 	99.97 	0 	99.97 	0.03 	0.79 	23.36
%	80000 	99.97 	0 	99.97 	0.03 	0.85 	26.2
%	100000 	99.97 	0 	99.97 	0.03 	0.97 	31.56
%	125000 	99.97 	0 	99.97 	0.03 	1.11 	37.7
%	200000 	99.96 	0 	99.96 	0.04 	1.52 	52.89
%	300000 	99.94 	0 	99.94 	0.06 	2.05 	67.33
%	400000 	99.91 	0 	99.91 	0.09 	2.62 	77.08
%	500000 	99.87 	0 	99.87 	0.13 	3.24 	83.67
%	
	% L = 2
	1000 	16.66 	21.23 	37.89 	62.11 	10.18 	0.38
	5000 	59.11 	11.17 	70.28 	29.72 	8.24 	1.89
	10000 	82.54 	5.13 	87.67 	12.33 	6.23 	3.74
	15000 	92.24 	2.39 	94.63 	5.37 	4.68 	5.55
	20000 	96.42 	1.18 	97.6 	2.4 	3.45 	7.33
	30000 	99.14 	0.29 	99.43 	0.57 	1.98 	10.8
	50000 	99.9 	0.02 	99.92 	0.08 	1.05 	17.33
	70000 	99.94 	0.0 	99.94 	0.06 	1.13 	23.36
	80000 	99.94 	0.0 	99.94 	0.06 	1.21 	26.2
	100000 	99.94 	0.0 	99.94 	0.06 	1.37 	31.56
	125000 	99.93 	0.0 	99.93 	0.07 	1.57 	37.7
	200000 	99.91 	0.0 	99.91 	0.09 	2.14 	52.89
	300000 	99.87 	0.0 	99.87 	0.13 	2.91 	67.33
	400000 	99.82 	0.0 	99.82 	0.18 	3.71 	77.08
	500000 	99.75 	0.0 	99.75 	0.25 	4.58 	83.67

\end{filecontents*}

\begin{filecontents*}[overwrite]{Cuckoo.dat}
	w	 FPR 	 FNR 	 Error 	 1-E 	 Phi 	 percentDup
	100 	30.52 	18.26 	48.78 	51.22 	2.17 	0.04
	500 	30.45 	19.11 	49.56 	50.44 	4.78 	0.19
	1000 	30.35 	19.98 	50.33 	49.67 	6.63 	0.38
	5000 	29.79 	30.35 	60.14 	39.86 	11.77 	1.89
	10000 	29.34 	38.26 	67.6 	32.4 	13.34 	3.74
	15000 	29.06 	43.68 	72.74 	27.26 	13.55 	5.55
	20000 	28.83 	47.68 	76.51 	23.49 	13.29 	7.33
	30000 	28.55 	53.19 	81.75 	18.25 	12.3 	10.8
	50000 	28.4 	59.14 	87.53 	12.47 	10.24 	17.33
	70000 	28.31 	62.16 	90.47 	9.53 	8.76 	23.36
	80000 	28.33 	63.14 	91.47 	8.53 	8.14 	26.2
	100000 	28.31 	64.59 	92.9 	7.1 	7.17 	31.56
	125000 	28.31 	65.75 	94.06 	5.94 	6.25 	37.7
	200000 	28.3 	67.46 	95.76 	4.24 	4.6 	52.89
	300000 	28.29 	68.38 	96.68 	3.32 	3.38 	67.33
	400000 	28.27 	68.8 	97.07 	2.93 	2.67 	77.08
	500000 	28.29 	69.02 	97.31 	2.69 	2.16 	83.67
\end{filecontents*}

\begin{filecontents*}[overwrite]{Queueing_SQF.dat}
	w	 FPR 	 FNR 	 Error 	 1-E 	 Phi 	 percentDup
	100 	0.26 	15.6 	15.86 	84.14 	30.54 	0.04
	500 	1.26 	17.48 	18.74 	81.26 	30.03 	0.19
	1000 	2.48 	17.31 	19.79 	80.21 	29.96 	0.38
	5000 	11.08 	19.93 	31. 	69. 	28.5 	1.89
	10000 	19.52 	22.38 	41.91 	58.09 	26.73 	3.74
%	15000 	26.04 	24.33 	50.37 	49.63 	25.1 	5.55
%	20000 	31.16 	25.78 	56.93 	43.07 	23.64 	7.33
%	30000 	38.39 	28.06 	66.45 	33.55 	21.09 	10.8
%	50000 	46.22 	30.96 	77.18 	22.82 	17.27 	17.33
%	70000 	49.88 	32.95 	82.83 	17.17 	14.58 	23.36
%	80000 	50.96 	33.66 	84.62 	15.38 	13.6 	26.2
%	100000 	52.35 	34.86 	87.21 	12.79 	11.98 	31.56
%	125000 	53.28 	35.99 	89.26 	10.74 	10.52 	37.7
%	200000 	54.22 	37.86 	92.08 	7.92 	8.02 	52.89
%	300000 	54.26 	38.94 	93.2 	6.8 	6.48 	67.33
%	400000 	53.77 	39.51 	93.27 	6.73 	5.75 	77.08
%	500000 	52.81 	39.83 	92.64 	7.36 	5.53 	83.67
%	
	% L=2
	15000 	13.43 	30.79 	44.21 	55.79 	34.4 	5.55
	20000 	16.62 	32.34 	48.96 	51.04 	33.03 	7.33
	30000 	21.7 	34.83 	56.53 	43.47 	30.61 	10.8
	50000 	28.33 	38.48 	66.81 	33.19 	26.5 	17.33
%	70000 	32.23 	41.05 	73.28 	26.72 	23.24 	23.36
%	80000 	33.57 	42.09 	75.66 	24.34 	21.85 	26.2
%	100000 	35.49 	43.73 	79.22 	20.78 	19.57 	31.56
%	125000 	37 	45.26 	82.26 	17.74 	17.33 	37.7
%	200000 	39.02 	47.9 	86.91 	13.09 	13.11 	52.89
%	300000 	39.81 	49.55 	89.36 	10.64 	10 	67.33
%	400000 	39.83 	50.27 	90.1 	9.9 	8.33 	77.08
%	500000 	39.47 	50.72 	90.19 	9.81 	7.26 	83.67
%	
	%L=1
	70000 	14.16 	59.09 	73.25 	26.75 	28.08 	23.36
	80000 	15.11 	59.81 	74.92 	25.08 	26.76 	26.2
	100000 	16.6 	60.89 	77.49 	22.51 	24.6 	31.56
	125000 	17.97 	62. 	79.96 	20.04 	22.27 	37.7
	200000 	20.36 	64.01 	84.36 	15.64 	17.27 	52.89
	300000 	21.76 	65.35 	87.11 	12.89 	13.14 	67.33
	400000 	22.51 	65.87 	88.38 	11.62 	10.52 	77.08
	500000 	22.86 	66.16 	89.02 	10.98 	8.7 	83.67
\end{filecontents*}

\begin{filecontents*}[overwrite]{SQF.dat}
	w	 FPR 	 FNR 	 Error 	 1-E 	 Phi 	 percentDup
	100 	34.26 	0.11 	34.37 	65.63 	2.69 	0.04
	500 	34.17 	0.55 	34.72 	65.28 	6 	0.19
	1000 	34.04 	1.06 	35.11 	64.89 	8.4 	0.38
	5000 	33.13 	5.38 	38.51 	61.49 	17.62 	1.89
	10000 	32.14 	10.21 	42.35 	57.65 	23.03 	3.74
	15000 	31.28 	14.46 	45.74 	54.26 	26.18 	5.55
	20000 	30.54 	18.27 	48.81 	51.19 	28.11 	7.33
	30000 	29.33 	24.73 	54.06 	45.94 	30.03 	10.8
	50000 	27.73 	34.43 	62.16 	37.84 	30.17 	17.33
	70000 	26.79 	41.1 	67.89 	32.11 	28.62 	23.36
	80000 	26.46 	43.67 	70.13 	29.87 	27.67 	26.2
	100000 	25.99 	47.72 	73.71 	26.29 	25.74 	31.56
	125000 	25.63 	51.4 	77.03 	22.97 	23.46 	37.7
	200000 	25.14 	57.56 	82.71 	17.29 	18.19 	52.89
	300000 	24.91 	61.16 	86.07 	13.93 	13.76 	67.33
	400000 	24.76 	62.88 	87.64 	12.36 	10.95 	77.08
	500000 	24.58 	63.82 	88.4 	11.6 	9.04 	83.67
\end{filecontents*}

\begin{filecontents*}[overwrite]{Queueing_StableBloom.dat}
	w	 FPR 	 FNR 	 Error 	 1-E 	 Phi 	 percentDup
	100 	2.01 	15.13 	17.14 	82.86 	11.41 	0.04
	500 	9.5 	15.57 	25.07 	74.93 	11.07 	0.19
%	1000 	17.94 	13.63 	31.57 	68.43 	10.9 	0.38
%	5000 	50.27 	14.79 	65.06 	34.94 	9.51 	1.89
%	10000 	56.45 	24.25 	80.7 	19.3 	7.4 	3.74
%	15000 	57.71 	29.07 	86.79 	13.21 	6.14 	5.55
%	20000 	58.29 	31.66 	89.96 	10.04 	5.32 	7.33
%	30000 	58.81 	34.32 	93.12 	6.88 	4.35 	10.8
%	50000 	59.24 	36.4 	95.64 	4.36 	3.37 	17.33
%	70000 	59.39 	37.36 	96.75 	3.25 	2.81 	23.36
%	80000 	59.45 	37.59 	97.05 	2.95 	2.65 	26.2
%	100000 	59.43 	38.02 	97.46 	2.54 	2.42 	31.56
%	125000 	59.47 	38.36 	97.83 	2.17 	2.15 	37.7
%	200000 	59.23 	38.89 	98.13 	1.87 	1.91 	52.89
%	300000 	58.81 	39.16 	97.98 	2.02 	1.94 	67.33
%	400000 	57.92 	39.41 	97.33 	2.67 	2.29 	77.08
%	500000 	56.83 	39.54 	96.37 	3.63 	2.74 	83.67
%	
	% L = 2
	1000 	11.16 	22.29 	33.45 	66.55 	12.86 	0.38
	5000 	34.88 	23.64 	58.52 	41.48 	11.78 	1.89
	10000 	41.03 	34.42 	75.45 	24.55 	9.43 	3.74
	15000 	42.63 	40.38 	83.01 	16.99 	7.85 	5.55
	20000 	43.37 	43.47 	86.84 	13.16 	6.9 	7.33
%	30000 	44.15 	46.77 	90.92 	9.08 	5.66 	10.8
%	50000 	44.75 	49.46 	94.2 	5.8 	4.4 	17.33
%	70000 	44.96 	50.74 	95.7 	4.3 	3.65 	23.36
%	80000 	45.04 	51.11 	96.15 	3.85 	3.39 	26.2
%	100000 	45.12 	51.6 	96.71 	3.29 	3.06 	31.56
%	125000 	45.17 	52.01 	97.18 	2.82 	2.74 	37.7
%	200000 	45.07 	52.71 	97.78 	2.22 	2.22 	52.89
%	300000 	44.8 	53.07 	97.86 	2.14 	2.01 	67.33
%	400000 	44.33 	53.28 	97.62 	2.38 	2.01 	77.08
%	500000 	43.66 	53.37 	97.03 	2.97 	2.2 	83.67
%	
	%L=1
	30000 	24.22 	64.19 	88.42 	11.58 	8.25 	10.8
	50000 	25.06 	67.47 	92.53 	7.47 	6.42 	17.33
	70000 	25.4 	68.95 	94.35 	5.65 	5.4 	23.36
	80000 	25.52 	69.42 	94.94 	5.06 	5.02 	26.2
	100000 	25.7 	70.13 	95.83 	4.17 	4.37 	31.56
	125000 	25.82 	70.62 	96.44 	3.56 	3.88 	37.7
	200000 	25.95 	71.44 	97.39 	2.61 	2.93 	52.89
	300000 	26.06 	71.87 	97.93 	2.07 	2.17 	67.33
	400000 	26.11 	72.07 	98.18 	1.82 	1.71 	77.08
	500000 	26.11 	72.19 	98.3 	1.7 	1.41 	83.67
	
\end{filecontents*}

\begin{filecontents*}[overwrite]{SBF.dat}
	w	 FPR 	 FNR 	 Error 	 1-E 	 Phi 	 percentDup
	100 	27.63 	0 	27.63 	72.37 	3.15 	0.04
	500 	27.51 	0 	27.51 	72.49 	7.07 	0.19
	1000 	27.38 	0.03 	27.41 	72.59 	9.97 	0.38
	5000 	26.52 	13.16 	39.67 	60.33 	18.34 	1.89
	10000 	26.32 	37.42 	63.74 	36.26 	15.37 	3.74
	15000 	26.3 	49.13 	75.43 	24.57 	12.58 	5.55
	20000 	26.3 	55.02 	81.32 	18.68 	10.88 	7.33
	30000 	26.3 	61.06 	87.36 	12.64 	8.77 	10.8
	50000 	26.28 	65.8 	92.08 	7.92 	6.7 	17.33
	70000 	26.3 	67.86 	94.15 	5.85 	5.53 	23.36
	80000 	26.3 	68.49 	94.79 	5.21 	5.12 	26.2
	100000 	26.29 	69.37 	95.65 	4.35 	4.52 	31.56
	125000 	26.31 	70.09 	96.4 	3.6 	3.91 	37.7
	200000 	26.25 	71.12 	97.37 	2.63 	2.94 	52.89
	300000 	26.27 	71.66 	97.93 	2.07 	2.17 	67.33
	400000 	26.25 	71.93 	98.18 	1.82 	1.71 	77.08
	500000 	26.23 	72.06 	98.28 	1.72 	1.42 	83.67
\end{filecontents*}



\begin{tikzpicture}
\begin{semilogxaxis}[legend pos=north west, xlabel=$w$, ylabel=$\FPR^w + \FNR^w (\times 100)$]
%  	\addplot table[x=a,y=b] {10e4.dat};
%\addlegendentry{$10^4$}
%\addplot[color=red, mark=+] table[x=w,y=Error] {Queueing_QHT.dat};
%\addlegendentry{Q\_QHT};
%\addplot[color=red, mark=+, dashed] table[x=w,y=Error] {QHT.dat};
%\addlegendentry{QHT};

\addplot[color=blue, mark=x] table[x=w,y=Error] {Queueing_SQF.dat};
\addlegendentry{Q\_SQF};
\addplot[color=blue, mark=x, dashed] table[x=w,y=Error] {SQF.dat};
\addlegendentry{SQF};
%%
\addplot[color=teal, mark=triangle] table[x=w,y=Error] {Queueing_Cuckoo.dat};
\addlegendentry{Q\_Cuckoo};
\addplot[color=teal, mark=triangle, dashed] table[x=w,y=Error] {Cuckoo.dat};
\addlegendentry{Cuckoo};
%%
\addplot[color=olive, mark=square, mark options = {scale=0.7}] table[x=w,y=Error] {Queueing_StableBloom.dat};
\addlegendentry{$Q\_SBF$};
\addplot[color=olive, mark=square, mark options = {scale=0.7}, dashed] table[x=w,y=Error] {SBF.dat};
\addlegendentry{$SBF$};

%\addplot[mark=x,^8$}
\end{semilogxaxis}

\end{tikzpicture}
	\caption{Comparing performances of QHT and SQF filters, in `vanilla' setting or when placed in our queueing structure.}\label{fig:improvement}
\end{figure}

We observe that queueing filters do not necessarily behave better than their 'vanilla' counterparts, especially on large sliding windows. This can be interpreted by the fact that the DDPs were optimised for infinite sliding windows, and as such operate better than their queueing equivalent on large sliding windows.

\section{Adversarial Resistance of Queueing Filters}
\label{sec:adversarial}

As DDFs have numerous security applications, we now discuss the queuing construction from an adversarial standpoint. We consider an adversarial game in which the attacker wants to trigger false positives or false negatives over the sliding window. One motivation for doing so is causing cache saturation or denial of service by forcing cache misses, triggering false alarms or crafting fradulent transactions without triggering fraud detection systems.

In order to create a realistic adversary model, we assume like in \cite{DBLP:journals/corr/abs-1709-08920} that the adversary does not have access to the internal memory representation of the filter. Nonetheless, after every insertion she knows whether the inserted element was detected as a duplicate or not. 

We first recall the definition of an adversarial game, adapted to our context.
\begin{definition}
	An adversary $\mathcal A$ feeds data to a sliding window DDF $\mathcal Q$, and for each inserted element, $\mathcal A$ knows whether $\mathcal Q$ answers $\DUPLICATE$ or $\UNSEEN$, but has not access to $\mathcal Q$'s internal state $\mathcal M$. The game has two distinct parts.
\begin{itemize}
	\item In the first part, $\mathcal A$ can feed up to $n$ elements to $\mathcal Q$ and learn $\mathcal Q$'s response for each insertion.
	\item In the second part, $\mathcal A$ sends a unique element $e^\star$.
\end{itemize} 
 $\mathcal A$ wins the $n$-false positive adversarial game (resp. $n$-false negative adversarial game) if and only if $e^\star$ is a false positive (resp. a false negative).
 
 \end{definition}
Variants of these games over a sliding window of size $w$ are immediate.
 
\subsubsection{False positive and false negative resistance}

\begin{definition}[Adversarial False Positive Resistance]
	We say that a DDF $\mathcal F$ is $(p, n)$-\emph{resistant to adversarial false positives} if no polynomial-time probabilistic (PPT) adversary $\mathcal A$ can win the $n-$false positive adversarial game with probability more than $p$.  
\end{definition}
Note that if $\mathcal F$ is $(p,n)$-resistant, then it is $(p, m)$-resistant for all $m < n$.

We define similarly the notion of being \emph{resistant to adversarial false negatives}. Finally, both definitions also make sense in a sliding window context.

%\paragraph{\textbf{Bound on false positive resistance}}

\begin{restatable}[Bound on false positive resistance]{thm}{fprRes}
	Let $\mathcal Q$ be a filter of $L$ subfilters $\mathcal F_i$, with $c$ insertions maximum per subfilter, let $w$ be a sliding window. 
	
	If $\mathcal F$ is $(p, c)$-resistant to adversarial false positive attacks and $cL \leq w$, then $\mathcal Q$ is $(1-(1-p)^L, w)$-resistant to adversarial false positive attacks on a sliding window of size $w$.
	
	If $cL > w$, the adversary has a probability of success of at least $1-(1-p)^L$.
\end{restatable}

%\begin{proof}
%	If $cL \leq w$, then information-theoretically the subfilters only have information on elements in the sliding window.
%	The probability of false positive for $\mathcal Q$ is $1 - (1 - \FP_{\mathcal F, c})^L$, which is strictly increasing with $\FP_{\mathcal F,c}$. Hence, the optimal solution is reached by to maximising the probability of false positive in each subfilter $\mathcal F_i$. By hypothesis the latter is bounded above by $p$ after $c$ insertions.
%	
%	On the other hand, if $cL > w$ then oldest filter holds information about elements that are not in the sliding window anymore. Hence, a strategy for the attacker trying to trigger a false positive on $e^\star$ could be to make it so these oldest elements are all equal to $e^\star$. Let $E$ be the optimal adversarial stream for triggering a false positive on the sliding window $w$ with the element $e^\star$, when $cL \leq w$. The adversary $\mathcal A$ can create a new stream $E' = e^\star | e^\star | \dotsc | E$ where $e^\star$ is prepended $cL - w$ times to $E$.
%	
%	After $w$ insertions, the last subfilter will answer $\DUPLICATE$ with probability at least $p$, hence giving a lower bound on $\mathcal A's$ probability of success. If, for some reason, the last subfilter answers $\DUPLICATE$ with probability less than $p$, then the same reasoning as for when $cL \leq w$ still applies, hence we get the correspondig lower bound (which is, in this case, an equality).
%\end{proof}
%%
%\paragraph{\textbf{Bounds on false negative resistance}}

\begin{restatable}[Bounds on false negative resistance]{thm}{fnrRes}
	Let $\mathcal Q$ be a filter of $L$ subfilters of kind $\mathcal F$, with $c$ insertions maximum per subfilter, and let $w$ be a sliding window. 
	
	If $\mathcal F$ is $(p, c)$-resistant to adversarial false negative attacks, then $\mathcal A$ can win the adversarial game on the sliding window $w$ with probability at least $p^L$.
	
	Furthermore, for $q$ the lower bound on the probability of false positive $\FP_{\mathcal F,c}$ for a given stream, if $w \leq (L-1)c$ then $\mathcal Q$ is $(\min(1 - q, p)^{L-1}p, w)$-resistant to false negative attacks on the sliding window $w$.
	On the other hand if $w > (L-1)c$ then $\mathcal Q$ is  $(\max(1 - q, p)^{L}, w)$-resistant to false negative attacks on the sliding window $w$.
\end{restatable}

%\begin{proof}
%	Let us first prove that a PPT adversary $\mathcal A$ can win the game with probability at least $p^L$.
%	For this, let us consider the adversarial game against the subfilter $\mathcal F$: after $c$ insertions from an aversarial stream $E_c$, $\mathcal A$ choses a duplicate $e^\star$ which will be a false negative with proability $p$.
%	Hence, if $\mathcal A$ crafts, for the filter $\mathcal Q$, the following adversarial stream $E' = E_c \mid E_c \mid \cdots \mid E_c$ consisting of $L$ concatenations of the stream $E_c$, then $e^\star$ is a false negative for $\mathcal Q$ if and only if it is a false negative for all subfilters $\mathcal F_i$, hence a probability of success for $\mathcal A$ of $p^L$.
%	
%	Now, Let us prove the case where $w\leq (L-1)c$. In this case, at any time, $\mathcal Q$ remembers all elements from inside the sliding window. As we have seen in the previous example, the probability of success of $\mathcal A$ is strictly increasing with the probability of each subfilter to answer $\UNSEEN$. The probability of a subfilter to answer $\UNSEEN$ is:
%	
%	\begin{itemize}
%		\item $\FN'_{\mathcal F,c}$ if $e^\star$ is in the subfilter's sub-sliding window;
%		\item $1 - \FP'_{\mathcal F, c}$ if $e^\star$ is not in the subfilter's sub-sliding window
%	\end{itemize}
%	where $\FN'$ and $\FP'$ are the probabilities of false negative and positives on the adversarial stream (which may be different from a random uniform stream).
%	
%	However, since $e^\star$ is a duplicate, it is in at least one subfilter's sub-sliding window. As such, the optimal strategy for $\mathcal A$ is to maximise the probability of all subfilters to answer $\UNSEEN$.  
%	Now, $\FN'_{\mathcal F,c}$ is bounded above by $p$ and $1 - \FP'_{\mathcal F, c}$ is bounded above by $1-q$, so the best strategy is where as many filters as possible answer $\UNSEEN$ with probability $\max(p, 1-q)$, knowing that at least one filter must contain $e^\star$ and as such its probability for returning $\UNSEEN$ is at most $p$, hence the result.
%	
%	Now, let us consider the case when $w > (L-1)c$. We have already introduce the element $e^\star$ in the last $w$ elements, and we want to insert it again. It is possible, for the adversary, to create the following stream $E = (e_1, e_2, \dotsc, e_{c-1}, e^\star, e_{c+1}, \dotsc, e_{Lc}, e_{Lc + 1})$, and to insert $e^\star$ afterwards.
%	
%	When $e_{Lc+1}$ is inserted, all elements $(e_1, \dotsc, e_{c-1}, e^\star)$ are dropped as the oldest subfilter is popped. Hence, in this context $e^\star$ is not in any subfilter anymore, so by adapting the previous analysis, $\mathcal A$ can get a false negative with probability at most $\max(1 - q, p)^{L}$.
%\end{proof}
%% Acknowledgements 
%\begin{acks} \end{acks}

%% Bibliography

