\documentclass[sigconf]{acmart}


\def\BibTeX{{\rm B\kern-.05em{\sc i\kern-.025em b}\kern-.08emT\kern-.1667em\lower.7ex\hbox{E}\kern-.125emX}}
 
\copyrightyear{2018}
\acmYear{2018}
\setcopyright{acmlicensed}
\acmConference[Woodstock '18]{Woodstock '18: ACM Symposium on Neural Gaze Detection}{June 03--05, 2018}{Woodstock, NY}
\acmBooktitle{Woodstock '18: ACM Symposium on Neural Gaze Detection, June 03--05, 2018, Woodstock, NY}
\acmPrice{15.00}
\acmDOI{10.1145/1122445.1122456}
\acmISBN{978-1-4503-9999-9/18/06}

%
% These commands are for a JOURNAL article.
%\setcopyright{acmcopyright}
%\acmJournal{TOG}
%\acmYear{2018}\acmVolume{37}\acmNumber{4}\acmArticle{111}\acmMonth{8}
%\acmDOI{10.1145/1122445.1122456}

%
% Submission ID. 
% Use this when submitting an article to a sponsored event. You'll receive a unique submission ID from the organizers
% of the event, and this ID should be used as the parameter to this command.
%\acmSubmissionID{123-A56-BU3}

%
% The majority of ACM publications use numbered citations and references. If you are preparing content for an event
% sponsored by ACM SIGGRAPH, you must use the "author year" style of citations and references. Uncommenting
% the next command will enable that style.
%\citestyle{acmauthoryear}


\begin{document}


\title{Adapting Duplicate Filters to a Sliding Window Context}

%
% The "author" command and its associated commands are used to define the authors and their affiliations.
% Of note is the shared affiliation of the first two authors, and the "authornote" and "authornotemark" commands
% used to denote shared contribution to the research.
\author{R\'emi G\'eraud-Stewart}
\affiliation{%
  \institution{D\'epartement d'informatique de l'ENS, \'Ecole normale sup\'erieure, CNRS, PSL Research University}
  \city{Paris}
  \state{France}
}
\affiliation{%
	\institution{Ingenico}
	\city{Paris}
	\state{France}
}
\email{remi.geraud@ens.fr}

\author{Marius Lombard-Platet}
\affiliation{%
  \institution{D\'epartement d'informatique de l'ENS, \'Ecole normale sup\'erieure, CNRS, PSL Research University}
  \city{Paris}
  \country{France}
}
\affiliation{%
\institution{Be-Studys}
\city{Geneva}
\country{Switzerland}
}
\email{marius.lombard-platet@ens.fr}

\author{David Naccache}
\affiliation{%
  \institution{D\'epartement d'informatique de l'ENS, \'Ecole normale sup\'erieure, CNRS, PSL Research University}
  \city{Paris}
  \country{France}
}
\affiliation{%
	\institution{Ingenico}
	\city{Paris}
	\state{France}
}
\email{david.naccache@ens.fr}

%
% By default, the full list of authors will be used in the page headers. Often, this list is too long, and will overlap
% other information printed in the page headers. This command allows the author to define a more concise list
% of authors' names for this purpose.
\renewcommand{\shortauthors}{Trovato and Tobin, et al.}

\begin{abstract}
We introduce a generic framework for adapting duplicate filters for sliding windows. Most papers on the literature focus on the fact that a filter must detect a duplicate on the whole stream. In this paper, we recall from previous work that this problem is ill-defined, and that


TODO 
\end{abstract}

%
% The code below is generated by the tool at http://dl.acm.org/ccs.cfm.
% Please copy and paste the code instead of the example below.
%
\begin{CCSXML}

\end{CCSXML}


\keywords{datasets, neural networks, gaze detection, text tagging}


\maketitle

\section{Introduction}
As the Internet has grown in popularity, data streams have become bigger and bigger. However, while the datasets grow, the need for a quick answer has remained the same. Yet, in many cases, it is not possible to find the optimal solution by storing all the data, may it be for time limitation or memory limitations. Hence, researchers have gone forward and proposed approximate algorithms, that thrive to get as close as possible to the solution, despite the lack of memory and CPU time. 

Such algorithms are present in network management \cite{10.5555/647912.740658}, in credit card fraud \cite{DBLP:journals/corr/abs-1709-08920}, phone calls data mining \cite{10.1145/347090.347094} and many other topics. A discussion about algorithms on large data streams can be found in \cite{10.1145/776985.776986}. 

In this paper, we focus on the topic of duplicate detection. While many authors in the literature tackle the problem of approximate duplicate detection with fuzzy matching, \cite{10.5555/1287369.1287420,1410199,10.1109/ICDE.2011.5767865,10.1109/ICDE.2012.20,Monge97anefficient}, we instead focus on \emph{exact} duplicate detection, in which an element is a duplicate if and only if it has already appeared in the stream.

As it turns out, (exact) duplicate detection has many real-life use cases, and can sometimes play a critical role, for instance in cryptographic schemes where all security and secrecy fall apart as soon as a random nonce is used twice, such as the Elgamal scheme \cite{10.1007/3-540-39568-7_2}. Other uses include improvements over caches \cite{4484874}, duplicate clicks \cite{10.1145/1060745.1060753}, and others.



%\begin{acks}
%
%\end{acks}

%
% The next two lines define the bibliography style to be used, and the bibliography file.
\bibliographystyle{ACM-Reference-Format}
\bibliography{sample-base}

% 
% If your work has an appendix, this is the place to put it.
%\appendix

%\section{Research Methods}

\end{document}
